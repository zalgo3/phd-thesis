\title{\TheTitle}
\author{\TheAuthor}
\date{\TheDate}

\usepackage{ifdraft}
\usepackage{ifthen}

\ifdraft{
    \usepackage[style=draft,maxnames=99,giveninits]{biblatex}
    \usepackage{biblatex-readbbl}
}{
    \usepackage[maxnames=99,giveninits]{biblatex}
    \usepackage{biblatex-readbbl}
}

\usepackage{graphicx}
\usepackage{adjustbox}
\usepackage{subcaption}

\usepackage[chapter]{algorithm}
\usepackage{algpseudocode}

\usepackage{booktabs}
\usepackage{tabularray}
\usepackage[l3]{csvsimple}

\usepackage{amsmath,amssymb}
\usepackage{cases}
\usepackage{derivative}
\usepackage[ntheorem]{empheq}
\usepackage{mathtools}
\mathtoolsset{showonlyrefs}
\usepackage[amsmath,amsthm,thmmarks]{ntheorem}
\usepackage{zref-titleref}
\usepackage{zref-clever}
\usepackage{hyperref}
\usepackage{mleftright}
\usepackage{mathrsfs}


\usepackage[top=4cm, bottom=4cm, left=3cm, right=3cm]{geometry}
\usepackage{setspace}
\usepackage{tocloft}
\usepackage{fancyhdr}
\usepackage{csquotes}
\usepackage[english]{babel}
\usepackage[intoc]{nomencl}
\usepackage{subfiles}

\usepackage{siunitx}
\usepackage[inline]{enumitem}


\def\usebbsetcapital{\def\setcapital##1{\mathbb{##1}}}
\def\usebfsetcapital{\def\setcapital##1{\mathbf{##1}}}
\def\usebmsetcapital{\def\setcapital##1{\bm{##1}}}
\usebfsetcapital
\def\setC{\setcapital{C}}
\def\setH{\setcapital{H}}
\def\setN{\setcapital{N}}
\def\setNpos{\setcapital{N}_{\mathord{+}}}
\def\setP{\setcapital{P}}
\def\setQ{\setcapital{Q}}
\def\setR{\setcapital{R}}
\def\setRpos{\setcapital{R}_{\mathord{+}}}
\def\setZ{\setcapital{Z}}

\DeclareMathOperator*\argmax{argmax}
\DeclareMathOperator*\argmin{argmin}
\let\liminf\relax
\let\limsup\relax
\DeclareMathOperator*\liminf{liminf}
\DeclareMathOperator*\limsup{limsup}
\DeclareMathOperator*\conv{conv}
\DeclareMathOperator\dist{dist}
\DeclareMathOperator\interior{int}
\DeclareMathOperator\dom{dom}
\DeclareMathOperator\proj{\mathbf{proj}}
\DeclareMathOperator\prox{\mathbf{prox}}
\DeclareMathOperator\proxgrad{\mathcal{T}}
\DeclareMathOperator\envelope{\mathcal{M}}
\DeclareMathOperator\indicator{\delta}
\DeclareMathOperator\level{\mathbf{lev}}
\DeclareMathOperator\Beta{\mathrm{B}}
\DeclareMathOperator\simplex{\Delta}
\DeclarePairedDelimiter\abs{\lvert}{\rvert}
\DeclarePairedDelimiter\norm{\lVert}{\rVert}
\DeclarePairedDelimiter\ceil{\lceil}{\rceil}
\DeclarePairedDelimiter\floor{\lfloor}{\rfloor}
\DeclarePairedDelimiter\set{\lbrace}{\rbrace}
\DeclarePairedDelimiterX\Set[2]{\lbrace}{\rbrace}{#1\mathrel{}\delimsize\vert\mathrel{}#2}
\DeclarePairedDelimiterX\innerp[2]{\langle}{\rangle}{#1, #2}

\newcommand\condition[1]{\quad \text{#1}}
\newcommand\forallcondition[1]{\condition{for all~$#1$}}
\newcommand\eqand{\quad \text{and} \quad}
\newcommand\st{\mathrm{s.t.}}
\newcommand\otherwise{\text{otherwise}}
\newcommand\T{\top\hspace{-1pt}}
\newcommand\acc{\mathrm{acc}}
\renewcommand\jdv[2]{\mathcal{J}_{#1}(#2)}


\AtBeginEnvironment{bmatrix}{\everymath{\displaystyle}}
\AtBeginEnvironment{pmatrix}{\everymath{\displaystyle}}


\makenomenclature
\renewcommand\nomname{List of Symbols and Notations}
\renewcommand{\nomgroup}[1]{
    \ifthenelse{\equal{#1}{A}}{\item[\textbf{Vector Spaces}]}{}
    \ifthenelse{\equal{#1}{B}}{\item[\textbf{Sets}]}{}
    \ifthenelse{\equal{#1}{C}}{\item[\textbf{Matrices}]}{}
    \ifthenelse{\equal{#1}{D}}{\item[\textbf{Functions and Operators}]}{}
}
\nomenclature[Aa]{$\setR$}{the set of real numbers}
\nomenclature[Aa]{$\setR^n$}{the $n$-dimensional real space}
\nomenclature[Aa]{$\setRpos^n$}{the nonnegative orthant in~$\setR^n$}
\nomenclature[Ab]{$\innerp{x}{y}$}{the Euclidean inner product between~$x$ and~$y$}
\nomenclature[Ab]{$\norm{x}_2$}{the $\ell_2$-norm of~$x$}
\nomenclature[Ab]{$\norm{x}_1$}{the $\ell_1$-norm of~$x$}
\nomenclature[Ab]{$\norm{x}_\infty$}{the $\ell_\infty$-norm of~$x$}
\nomenclature[Ab]{$(x, y)$}{the open line segment between~$x$ and~$y$}
\nomenclature[Ab]{$[x, y]$}{the closed line segment between~$x$ and~$y$}
\nomenclature[Ab]{$\dim(X)$}{the dimension of a space~$X$}
\nomenclature[Ba]{$\simplex^n$}{the unit $n$-simplex}
\nomenclature[Bb]{$\interior(C)$}{the interior of~$C$}
\nomenclature[Bb]{$\conv(C)$}{the convex hull of~$C$}
\nomenclature[Bc]{$\dist(x, C)$}{the distance between~$x$ and~$C$}
\nomenclature[C]{$\ker(A)$}{the kernel of a matrix~$A$}
\nomenclature[C]{$A^\T$}{the transpose of a matrix~$A$}
\nomenclature[C]{$I_n$}{the $n \times n$ identity matrix}
\nomenclature[Da]{$\nabla f(x)$}{the gradient of~$f$ at~$x$}
\nomenclature[Db]{$\jdv{f}{x}$}{the Jacobian matrix of~$f$ at~$x$}
\nomenclature[Db]{$\partial f(x)$}{the subdifferential of~$f$ at~$x$}
\nomenclature[Db]{$f'(x; d)$}{the directional derivative of~$f$ at~$x$ in a direction~$d$}
\nomenclature[Dc]{$\dom(f)$}{the effective domain of function~$f$}


\renewcommand\algorithmicrequire{\textbf{Input:}}
\renewcommand\algorithmicensure{\textbf{Output:}}

\renewcommand\baselinestretch{1.2}
\mleftright


\theoremclass{Theorem}
\theoremstyle{break}
\newtheorem{theorem}{Theorem}[chapter]
\newtheorem{proposition}{Proposition}[chapter]
\newtheorem{lemma}{Lemma}[chapter]
\newtheorem{corollary}{Corollary}[chapter]
\newtheorem{definition}{Definition}[chapter]
\newtheorem{example}{Example}[chapter]
\newtheorem{remark}{Remark}[chapter]
\newtheorem{assumption}{Assumption}[chapter]
\AtEndEnvironment{proof}{\qedsymbol{\Box}~\qed}


\zcsetup{
nameinlink,
}
\setlist[description]{style=nextline}
\setlist[enumerate,1]{
    label=(\roman*)
}
\setlist[enumerate,2]{
    label=(\alph*)
}
\zcRefTypeSetup{claim}{
    Name-sg = Claim,
    name-sg = claim,
    Name-pl = Claims,
    name-pl = claims,
}
\zcRefTypeSetup{equation}{
    name-sg = ,
    name-pl = ,
    +refbounds={,(,),},
}
\zcRefTypeSetup{assumption}{
    Name-sg = Assumption,
    name-sg = assumption,
    Name-pl = Assumptions,
    name-pl = assumptions,
}
\zcRefTypeSetup{theorem}{cap}
\zcRefTypeSetup{lemma}{cap}
\zcRefTypeSetup{proposition}{cap}
\zcRefTypeSetup{corollary}{cap}
\zcRefTypeSetup{definition}{cap}
\zcRefTypeSetup{example}{cap}
\zcRefTypeSetup{remark}{cap}
\zcRefTypeSetup{assumption}{cap}
\zcRefTypeSetup{algorithm}{cap}
\zcRefTypeSetup{chapter}{cap}
\zcRefTypeSetup{section}{cap}
\zcRefTypeSetup{table}{cap}
\zcRefTypeSetup{figure}{cap}
\makeatletter
\zref@newprop{theorem}{}
\zref@newprop{lemma}{}
\zref@newprop{proposition}{}
\zref@newprop{corollary}{}
\zref@newprop{definition}{}
\zref@newprop{example}{}
\zref@newprop{remark}{}
\zref@newprop{assumption}{}
\zref@newprop{thmtype}{}
\zref@addprops{main}{theorem,lemma,proposition,corollary,definition,example,remark,assumption,thmtype}
\BeforeBeginEnvironment{theorem}{
    \zref@setcurrent{theorem}{\thetheorem}
    \zref@setcurrent{thmtype}{theorem}
}
\BeforeBeginEnvironment{lemma}{
    \zref@setcurrent{lemma}{\thelemma}
    \zref@setcurrent{thmtype}{lemma}
}
\BeforeBeginEnvironment{proposition}{
    \zref@setcurrent{proposition}{\theproposition}
    \zref@setcurrent{thmtype}{proposition}
}
\BeforeBeginEnvironment{corollary}{
    \zref@setcurrent{corollary}{\thecorollary}
    \zref@setcurrent{thmtype}{corollary}
}
\BeforeBeginEnvironment{definition}{
    \zref@setcurrent{definition}{\thedefinition}
    \zref@setcurrent{thmtype}{definition}
}
\BeforeBeginEnvironment{example}{
    \zref@setcurrent{example}{\theexample}
    \zref@setcurrent{thmtype}{example}
}
\BeforeBeginEnvironment{remark}{
    \zref@setcurrent{remark}{\theremark}
    \zref@setcurrent{thmtype}{remark}
}
\BeforeBeginEnvironment{assumption}{
    \zref@setcurrent{assumption}{\theassumption}
    \zref@setcurrent{thmtype}{assumption}
}
\BeforeBeginEnvironment{algorithmic}{
    \zref@setcurrent{zc@type}{line}
}
\AfterEndEnvironment{proof}{
    \zref@setcurrent{thmtype}{}
}
\AtBeginEnvironment{enumerate}{
    \ifthenelse{\NOT \equal{\zref@getcurrent{thmtype}}{}}{
        \zref@setcurrent{zc@type}{claim}
    }{}
}
\newcommand\ChangeEnumRefProps[1]{%
    \ifthenelse{\equal{\zref@extract{#1}{zc@type}}{claim}}{%
        \zref@def@extract{\thmtype}{#1}{thmtype}% label's theorem type
        \zref@def@extract{\thethm}{#1}{\thmtype}% label's theorem number
        \zref@def@extract{\theclaimi}{#1}{default}% label's claim number
        \ifthenelse{\(\NOT \equal{\thmtype}{\zref@getcurrent{thmtype}}\) \OR \(\NOT \equal{\thethm}{\zref@getcurrent{\thmtype}}\)}{%
            \edef\mypatchzctype{\noexpand\patchcmd{\expandafter\noexpand\csname Z@R@#1\endcsname}{\noexpand\zc@type{claim}}{\noexpand\zc@type{\thmtype}}{}{}}%
            \edef\mypatchdefault{\noexpand\patchcmd{\expandafter\noexpand\csname Z@R@#1\endcsname}{\noexpand\default{\theclaimi}}{\noexpand\default{{\thethm~\theclaimi}}}{}{}}%
            \mypatchzctype%
            \mypatchdefault%
        }{}%
    }{}%
}
\let\savezcref\zcref
\ExplSyntaxOn
\NewDocumentCommand\newzcref{s O{} m}{%
    \seq_set_from_clist:Nn \l__newzcref_labels_seq {#3}%
    \seq_map_function:NN \l__newzcref_labels_seq \ChangeEnumRefProps%
    \IfBooleanTF{#1}{\savezcref*[#2]{#3}}{\savezcref[#2]{#3}}%
}
\ExplSyntaxOff
\let\zcref\newzcref
\makeatother

\hypersetup{
    colorlinks=true,
    allcolors=[HTML]{0645ad},
    pdftitle=\TheTitle,
    pdfauthor=\TheAuthor,
    final
}


\pagestyle{fancy}

\fancyhead[RE]{\leftmark} %section
\fancyhead[LE]{\thepage}
\fancyhead[LO]{\rightmark} % chapter
\fancyhead[RO]{\thepage}
\fancyfoot[C]{}
\setlength\headheight{28pt}


\makeatletter
\AtBeginDocument{
    \renewcommand\chapter{
        \if@openright\cleardoublepage\else\clearpage\fi
        \global\@topnum\z@
        \@afterindentfalse
    \secdef\@chapter\@schapter}
    \renewcommand\tableofcontents{
        \if@twocolumn
            \@restonecoltrue\onecolumn
        \else
            \@restonecolfalse
        \fi
        \chapter*{\contentsname
            \@mkboth{
        \MakeUppercase\contentsname}{\MakeUppercase\contentsname}}
        \@starttoc{toc}
        \if@restonecol\twocolumn\fi
    }
    \renewcommand\listoffigures{
        \if@twocolumn
            \@restonecoltrue\onecolumn
        \else
            \@restonecolfalse
        \fi
        \chapter*{\listfigurename}
        \@mkboth{\MakeUppercase\listfigurename}
        {\MakeUppercase\listfigurename}
        \@starttoc{lof}
        \if@restonecol\twocolumn\fi
    }
    \renewcommand\listoftables{
        \if@twocolumn
            \@restonecoltrue\onecolumn
        \else
            \@restonecolfalse
        \fi
        \chapter*{\listtablename}
        \@mkboth{
        \MakeUppercase\listtablename}
        {\MakeUppercase\listtablename}
        \@starttoc{lot}
        \if@restonecol\twocolumn\fi
    }
    \subfile{frontback/cover}
}
\makeatother

\AtEndDocument{
    \printbibliography
}


\addbibresource{library.bib}

\sloppy
