\documentclass[../main]{subfiles}

\begin{document}
\section{The multi-objective proximal PL inequality and error bounds} \zlabel{sec:merit:eb}
For the multi-objective composite problem~\zcref{eq:composite_MO}, this section extends the proximal-PL inequality introduced in \zcref{sec:preliminaries:PL} and shows that it induces the proposed merit function's error bound.
We first define the \emph{multi-objective proximal-PL inequality}.
\begin{definition}[Multi-objective proximal-PL inequality] \zlabel{def:proximal_PL_MO}
    Assume that~$f_i, i = 1, \dots, m$ is~$L_{f_i}$-smooth with~$L_{f_i} > 0$ and let~$L_f \coloneqq \max_{i = 1, \dots, m} L_{f_i}$.
    We say that~\zcref{eq:composite_MO} satisfies the multi-objective proximal-PL inequality if there exists~$\tau > 0$ such that
    \begin{equation} \label{eq:proximal_PL_MO}
        w_{L_f^{-1}}(x) \ge \tau u_\infty(x) \forallcondition{x \in \setR^n}
    \end{equation}
    with~$u_\infty$ and~$w_{L_f^{-1}}$ given by~\zcref{eq:gap_MO,eq:reg_lin_gap_MO}.
\end{definition}
If~$m = 1$, \zcref{eq:proximal_PL_MO} reduces to the proximal-PL inequality for scalar optimization~\zcref{eq:proximal_PL}.
We state below some sufficient conditions for~\zcref{eq:proximal_PL_MO}.
\begin{proposition} \zlabel{thm:proximal_PL_MO_sufficient}
    \begin{enumerate}
        \item When~$f_i$ is $\mu_{f_i}$-convex with~$\mu_{f_i} > 0$,~\zcref{eq:proximal_PL_MO} holds with~$\tau \coloneqq \min (\mu_f / L_f, 1)$, where~$\mu_f \coloneqq \min_{i = 1, \dots, m} \mu_{f_i}$. \zlabel{thm:proximal_PL_MO_sufficient:sc}
        \item Assume that~$f_i(x) \coloneqq h(A_i x)$ with some strongly convex function~$h_i$ and linear transformation~$A_i$, and~$g_i \coloneqq \indicator_{\mathcal{X}_i}$ with~$\mathcal{X}_i$ being a polyhedral set and~$\indicator_{\mathcal{X}_i}$ given by~\zcref{eq:indicator}.
            If each~$\min_{x \in \setR^n} F_i(x)$ has a nonempty set~$X_i^*$ for~$i = 1, \dots, m$, then~\zcref{eq:proximal_PL_MO} holds with some constant~$\tau$. \zlabel{thm:proximal_PL_MO_sufficient:sc_linear}
    \end{enumerate}
\end{proposition}
\begin{proof}
    \zcref[S]{thm:proximal_PL_MO_sufficient:sc}:
    Since~$f_i$ is strongly convex, \zcref{thm:merit_between:convex} gives
    \begin{equation}
        u_\infty(x) \le w_\mu(x) \forallcondition{x \in \setR^n}
    .\end{equation}
    Applying \zcref{thm:merit_inner} to the above inequality implies
    \begin{equation}
        u_\infty(x) \le \max \left( \frac{L_f}{\mu_f}, 1 \right) w_{L_f^{-1}}(x) \forallcondition{x \in \setR^n}
    ,\end{equation}
    which means
    \begin{equation}
        w_{L_f^{-1}}(x) \ge \min \left( \frac{\mu_f}{L_f}, 1 \right) u_\infty(x) \forallcondition{x \in \setR^n}
    .\end{equation}

    \zcref[S]{thm:proximal_PL_MO_sufficient:sc_linear}:
    Since~$\mathcal{X}_i$ is polyhedral, we can write it as~$\Set{x \in \setR^n}{B_i x \le c_i}$ for some matrix~$B_i$ and vector~$c_i$ for~$i = 1, \dots, m$.
    We now show that for all~$i = 1, \dots, m$ there exists some~$z_i$ such that
    \begin{equation}
        X_i^* = \Set{x \in \setR^n}{B_i x \le c_i \text{ and } A_i x = z_i}
    .\end{equation}
    To obtain a contradiction, suppose that there exists~$x^1 \in X_i^*$ and~$x^2 \in X_i^*$ such that~$A_i x^1 \neq A_i x^2$.
    Clearly, we have~$f_i(x^1) = f_i(x^2)$.
    Since~$h_i$ is strongly convex, we get
    \begin{align}
        f_i(x^1) &= \frac{1}{2} f_i(x^1) + \frac{1}{2} f_i(x^2) = \frac{1}{2} h_i(A_i x^1) + \frac{1}{2} h_i(A_i x^2) \\
                 &> h_i\left( A_i \left( \frac{1}{2} x^1 + \frac{1}{2} x^2 \right) \right) = f_i\left( \frac{1}{2} x^1 + \frac{1}{2} x^2 \right)  
    ,\end{align}
    which contradicts the fact that~$x^1 \in X_i^*$.
    Therefore, we can use Hoffman's error bound~\cite{Hoffman1952}, and so there exists some~$\rho_i > 0$ such that for any~$x \in \setR^n$, there exists~$x_i^* \in X_i^*$ with
    \begin{equation}
        \norm*{x - x_i^*}_2 \le \rho_i \norm*{\max \left[ \begin{pmatrix} B_i \\ A_i \\ - A_i \end{pmatrix} x - \begin{pmatrix} c_i \\ z_i \\ - z_i \end{pmatrix}, 0 \right] }_2
    .\end{equation}
    Note that we take the $\max$ operator componentwise on the right-hand side.
    Since~$B_i x - c_i \le 0$ for all~$x \in \dom(F)$, we have
    \begin{equation}
        \norm*{x - x_i^*}_2 \le \rho_i \norm*{\max \left[ \begin{pmatrix} A_i \\ - A_i \end{pmatrix} x - \begin{pmatrix} z_i \\ - z_i \end{pmatrix} , 0 \right] }_2 \forallcondition{x \in \dom(F)}
    ,\end{equation}
    which yields
    \begin{equation}
        \norm*{x - x_i^*}_2^2 \le \rho_i^2 \norm*{A_i x - z_i}_2^2 \forallcondition{x \in \dom(F)}
    .\end{equation}
    Since~$\proj_{X_i^*}(x) \in X_i^*$, it follows that
    \begin{equation} \label{eq:proximal_PL_MO_sufficient:sc_linear:linear_trans_ineq}
        \norm*{x - \proj_{X_i^*}(x)}_2^2 \le \norm*{x - x_i^*}_2^2 \le \rho_i^2 \norm*{A_i \left(x - \proj_{X_i^*}(x)\right)}_2^2 \forallcondition{x \in \dom(F)}
    .\end{equation}
    Now, suppose that~$x \in \dom(F)$.
    From the definition~\zcref{eq:gap_MO} of~$u_\infty$, we get
    \begin{align}
        u_\infty(x) &= \sup_{z \in \setR^n} \min_{i = 1, \dots, m} [F_i(x) - F_i(z)] \\
               &\le \min_{i = 1, \dots, m} \sup_{z \in \setR^n} [F_i(x) - F_i(z)] = \min_{i = 1, \dots, m} \left[ F_i(x) - F_i\left( \proj_{X_i^*}(x) \right) \right] 
    ,\end{align}
    where the second equality holds because~$\proj_{X_i^*}(x) = \argmin_{x \in \setR^n} F_i(x)$.
    Assuming that~$h_i$ is $\sigma_{h_i}$-convex with~$\sigma_{h_i} > 0$, it follows that
    \begin{align}
        u_\infty(x) &= \min_{i = 1, \dots, m}
        \begin{multlined}[t] \Biggl[ 
                \innerp*{\nabla h_i(A_i x)}{A_i \left(x - \proj_{X_i^*}(x)\right)} \\
                + g_i(x) - g_i\left( \proj_{X_i^*}(x) \right) - \frac{\sigma_{h_i}}{2} \norm*{A_i\left( x - \proj_{X_i^*}(x) \right) }_2^2 \Biggr] 
            \end{multlined} \\
               &= \min_{i = 1, \dots, m} 
               \begin{multlined}[t]
               \Biggl[ \innerp*{\nabla f_i(x)}{x - \proj_{X_i^*}(x)} + g_i(x) - g_i\left( \proj_{X_i^*}(x) \right) \\
               - \frac{\sigma_{h_i}}{2} \norm*{A_i \left( x - \proj_{X_i^*}(x) \right) }_2^2 \Biggr] 
               .\end{multlined}
    \end{align}
    Applying~\zcref{eq:proximal_PL_MO_sufficient:sc_linear:linear_trans_ineq} to the above inequality leads to
    \begin{equation}
        u_\infty(x) \le \min_{i = 1, \dots, m}
            \begin{multlined}[t]
                \Biggl[ \innerp*{\nabla f_i(x)}{x - \proj_{X_i^*}(x)} + g_i(x) - g_i\left( \proj_{X_i^*}(x) \right) \\
                    - \frac{\sigma_{h_i}}{2 \rho_i^2} \norm*{x - \proj_{X_i^*}(x)}_2^2 \Biggr]
            .\end{multlined}
    \end{equation}
    Let~$e \in \simplex^m$ with~$\simplex^m$ given by~\zcref{eq:simplex}.
    Since~$\min_{i = 1, \dots, m} v_i = \min_{e \in \simplex^m} \sum_{i = 1}^{m} e_i v_i$ for any~$v \in \setR^m$, we get
    \begin{align}
        u_\infty(x) &\le \min_{e \in \simplex^m} \sum_{i = 1}^{m} e_i
            \begin{multlined}[t]
                \Biggl[ \innerp*{\nabla f_i(x)}{x - \proj_{X_i^*}(x)} + g_i(x) - g_i\left( \proj_{X_i^*}(x) \right) \\
                    - \frac{\sigma_{h_i}}{2 \rho_i^2} \norm*{x - \proj_{X_i^*}(x)}_2^2 \Biggr]
            \end{multlined} \\
        &\le \min_{e \in \simplex^m} \sup_{z \in \setR^n} \sum_{i = 1}^{m} e_i \left[ \innerp*{\nabla f_i(x)}{x - z} + g_i(x) - g_i(z) - \frac{\sigma_{h_i}}{2 \rho_i^2} \norm{x - z}_2^2 \right] \\
        &= \sup_{z \in \setR^n} \min_{e \in \simplex^m} \sum_{i = 1}^{m} e_i \left[ \innerp*{\nabla f_i(x)}{x - z} + g_i(x) - g_i(z) - \frac{\sigma_{h_i}}{2 \rho_i^2} \norm{x - z}_2^2 \right] \\
        &= \sup_{z \in \setR^n} \min_{i = 1, \dots, m} \left[ \innerp*{\nabla f_i(x)}{x - z} + g_i(x) - g_i(z) - \frac{\sigma_{h_i}}{2 \rho_i^2} \norm{x - z}_2^2 \right] \\
        &\le w_{\rho_i^2 / \min\limits_{i = 1, \dots, m} \sigma_{h_i}}(x)
    ,\end{align}
    where the first equality follows from the Sion's minimax theorem~\cite{Sion1958}, and the third equality comes from the definition~\zcref{eq:reg_lin_gap_MO} of~$w_{\rho_i^2 / \min\limits_{i = 1, \dots, m} \sigma_{h_i}}$.
    Thus, \zcref{thm:merit_inner} gives
    \begin{equation}
        u_\infty(x) \le \max \left( \frac{L_f \rho_i^2}{\min\limits_{i = 1, \dots, m} \sigma_{h_i}} , 1 \right) w_{L_f^{-1}}(x)
    ,\end{equation}
    which completes the proof.
\end{proof}

We now show that the multi-objective proximal-PL inequality~\zcref{eq:proximal_PL_MO} leads to the error-bound property.
\begin{theorem} \zlabel{thm:proximal_PL_MO_EB}
    Let~$x \in \setR^n$.
    Suppose that~$f_i$ is~$L_{f_i}$-smooth with~$L_{f_i} > 0$ for each~$i = 1, \dots, m$,~$L_f \coloneqq \max_{i = 1, \dots, m} L_{f_i}$, and the multi-objective proximal-PL inequality~\zcref{eq:proximal_PL_MO} holds with~$\tau > 0$.
    Then, the trajectory~$\set*{W_{L_f^{-1}}^k(x) \coloneqq \overbrace{W_{L_f^{-1}} \circ \dots \circ W_{L_f^{-1}}}^{m}(x)}$ converges linearly to a weakly Pareto optimal point~$x^*$ and
    \begin{equation}
        u_\infty(x) \ge \frac{\tau L_f}{8} \norm{x - x^*}_2^2
    ,\end{equation}
    where~$u_\infty$ and~$W_{L_f^{-1}}$ are given by~\zcref{eq:gap_MO,eq:reg_lin_gap_MO_sol}, respectively.
\end{theorem}
\begin{proof}
    Recall that~$u_\infty$ is non-negative due to \zcref{thm:gap_MO}.
    We have
    \begin{equation}
        \sqrt{u_\infty(x)} - \sqrt{u_\infty\left(W_{L_f^{-1}}(x)\right)} = \frac{u_\infty(x) - u_\infty\left(W_{L_f^{-1}}(x)\right)}{\sqrt{u_\infty(x)} + \sqrt{u_\infty\left(W_{L_f^{-1}}(x)\right)}}
    .\end{equation}
    The definition~\zcref{eq:gap_MO} of~$u_\infty$ gives
    \begin{equation}
        u_\infty(x) - u_\infty\left(W_{L_f^{-1}}(x)\right) \ge \min_{i = 1, \dots, m} \left[F_i(x) - F_i\left(W_{L_f^{-1}}(x)\right)\right] \ge w_{L_f^{-1}}(x)
    ,\end{equation}
    where the second inequality follows from~\zcref{thm:descent,eq:reg_lin_gap_MO,eq:reg_lin_gap_MO_sol}.
    Note that this inequality, together with~\zcref{eq:proximal_PL_MO}, proves that~$\set*{W_{L_f^{-1}}^k(x)}$ converges linearly to zero.
    On the other hand, since~$u_\infty(x) \ge u_\infty\left(W_{L_f^{-1}}(x)\right)$ because of \zcref{thm:reg_lin_gap} and the above inequality, we get
    \begin{equation}
        \sqrt{u_\infty(x)} + \sqrt{u_\infty\left(W_{L_f^{-1}}(x)\right)} \le 2 \sqrt{u_\infty(x)} \le 2 \sqrt{w_{L_f^{-1}}(x) / \tau}
    ,\end{equation}
    where the second inequality comes from~\zcref{eq:proximal_PL_MO}.
    Then, the above three inequalities show
    \begin{equation}
        \sqrt{u_\infty(x)} - \sqrt{u_\infty\left(W_{L_f^{-1}}(x)\right)} \ge \frac{w_{L_f^{-1}}(x)}{2 \sqrt{w_{L_f^{-1}}(x) / \tau}} = \frac{1}{2} \sqrt{\tau w_{L_f^{-1}}(x)}
    .\end{equation}
    Therefore, it follows from~\zcref{eq:reg_lin_gap_MO_LB} that
    \begin{equation}
        \sqrt{u_\infty(x)} - \sqrt{u_\infty\left(W_{L_f^{-1}}(x)\right)} \ge \frac{\sqrt{\tau L_f}}{2 \sqrt{2}} \norm*{x - W_{L_f^{-1}}(x)}_2
    .\end{equation}
    More generally, we arrive at
    \begin{equation}
        \sqrt{u_\infty\left(W_{L_f^{-1}}^k(x)\right)} - \sqrt{u_\infty\left(W_{L_f^{-1}}^{k + 1}(x)\right)} \ge \frac{\sqrt{\tau L_f}}{2 \sqrt{2}} \norm*{W_{L_f^{-1}}^k(x) - W_{L_f^{-1}}^{k + 1}(x)}_2
    \end{equation}
    for all~$k = 0, 1, \dots$.
    Adding up the above inequality from~$k = k_1$ to~$k = k_2 - 1$ yields
    \begin{equation}
        \sqrt{u_\infty\left(W_{L_f^{-1}}^{k_1}(x)\right)} - \sqrt{u_\infty\left(W_{L_f^{-1}}^{k_2}(x)\right)} \ge \frac{\sqrt{\tau L_f}}{2 \sqrt{2}} \sum_{k = k_1}^{k_2 - 1} \norm*{W_{L_f^{-1}}^k(x) - W_{L_f^{-1}}^{k + 1}(x)}_2
    .\end{equation}
    Thus, the triangle inequality implies
    \begin{equation} \label{eq:proximal_PL_MO_EB:sq_gap_Cauchy}
        \sqrt{u_\infty\left(W_{L_f^{-1}}^{k_1}(x)\right)} - \sqrt{u_\infty\left(W_{L_f^{-1}}^{k_2}(x)\right)} \ge \frac{\sqrt{\tau L_f}}{2 \sqrt{2}} \norm*{W_{L_f^{-1}}^{k_1}(x) - W_{L_f^{-1}}^{k_2}(x)}_2
    .\end{equation}
    As~$k_1, k_2 \to \infty$, the left-hand side tends to zero.
    Therefore, the right-hand side also tends to zero because of the non-negativity of the norm.
    This means that~$\set{W_{L_f^{-1}}^k(x)}$ is the Cauchy sequence, which is convergent to some weakly Pareto optimal point~$x^*$.
    Substituting~$k_1 = 0$ and~$k_2 = \infty$ into~\zcref{eq:proximal_PL_MO_EB:sq_gap_Cauchy} leads to
    \begin{equation}
        \sqrt{u_\infty(x)} \ge \frac{\sqrt{\tau L_f}}{2 \sqrt{2}} \norm{x - x^*}_2
    .\end{equation}
\end{proof}
This theorem also presents the error-bound property of~$w_\alpha$ and~$u_\alpha$ for any~$\alpha > 0$ because of~\zcref{eq:proximal_PL_MO,thm:merit_inner,thm:merit_between:Lipschitz}.
\end{document}
