\documentclass[../main]{subfiles}

\begin{document}
\section{Introduction}
This chapter considers the convex-constrained multi-objective optimization problems, i.e.,~\zcref{eq:MOO} with~$C$ being non-empty, closed, and convex.
It presents new merit functions for them, and discusses their properties mentioned in \zcref{sec:intro:merit}.

In detail, it proposes the following three merit functions for~\zcref{eq:MOO}:
\begin{enumerate}
    \item the gap function for lower semi-continuous multi-objective optimization; \zlabel{enum:gap_MO}
    \item the regularized gap function for convex multi-objective optimization; \zlabel{enum:reg_gap_MO}
    \item the regularized and partially linearized gap function for composite multi-objective optimization. \zlabel{enum:reg_lin_gap_MO}
\end{enumerate}
In \zcref{tab:merit}, we summarize the properties of those merit functions, which will be shown in the subsequent sections.
There, `Sol.' represents the types of Pareto solutions for~\zcref{eq:MOO} corresponding to the merit functions' minima (zero points).
Moreover, `SP,' `LB,' and `EB' indicate each~$F_i$'s sufficient conditions so that stationary points of the merit functions can solve~\zcref{eq:MOO}, the merit functions are level-bounded, and the merit functions provide error bounds, respectively.
The gap function~\zcref[noname]{enum:gap_MO} connects its minima and the weak Pareto solutions of~\zcref{eq:MOO} but does not have good properties in other aspects.
The regularized gap function~\zcref[noname]{enum:reg_gap_MO} has better properties but requires the convexity of~$F_i$.
The regularized and partially linearized gap function~\zcref[noname]{enum:reg_lin_gap_MO} relaxes the convexity assumption and is easy to compute for particular problems.
\begin{table}[htpb]
    \caption{Properties of our proposed merit functions}
    \zlabel{tab:merit}
    \begin{subtable}{\textwidth}
        \centering
        \caption{Proposed merit functions and their properties}
        \begin{tblr}{hline{1,2,5}={solid}, hline{3,4}={dashed}, width=\textwidth, colspec={@{}X[.75,c]X[1.5,c]X[c]X[c]X[c]X[1.5,c]X[3,c]X[c]@{}}, rowspec={Q[m]Q[m]Q[m]Q[m]}}
                                                & Obj.  & Sol.                & Cont.                 & Diff.               & SP        & LB              & EB                  \\
            \zcref[noname]{enum:gap_MO}         & LSC   & \SetCell[r=2]{m}WPO & LSC                   & $\times$            & $\times$  & LB              & \SetCell[r=3]{m} PL \\
            \zcref[noname]{enum:reg_gap_MO}     & Conv. &                     & \SetCell[r=2]{m}Cont. & \SetCell[r=2]{m} DD & SC        & Conv., LB       &                     \\
            \zcref[noname]{enum:reg_lin_gap_MO} & Comp. & PS                  &                       &                     & SC, $C^2$ & Conv., LB, etc. &                     \\
        \end{tblr}
    \end{subtable}

    \bigskip
    \begin{subtable}{\textwidth}
        \centering
        \caption{Table of abbreviations}
        \begin{tblr}{@{}cc@{}}
            \hline
            Obj.  & Objective functions                    \\
            Sol.  & Solutions                              \\
            Cont. & Continuity                             \\
            Diff. & Differentiability                      \\
            SP    & Stationary points                      \\
            LB    & Level-boundedness                      \\
            EB    & Error bounds                           \\
            Cont. & Continuity                             \\
            Comp. & Composite                              \\
            WPO   & Weak Pareto optimality                 \\
            PS    & Pareto stationarity                    \\
            LSC   & Lower semicontinuity                   \\
            DD    & Directional differentiability          \\
            $C^2$ & Twice continuously differentiable      \\
            SC    & Strict convexity                       \\
            PL    & Multi-objective proximal PL inequality \\
            \hline
        \end{tblr}
    \end{subtable}
\end{table}

We summarize the structure of the rest of this chapter.
\zcref{sec:merit:merit} proposes different merit functions for multi-objective optimization with loser semi-continuous, convex, and composite objectives, respectively, along with methods for evaluating the function values, the differentiability, and the stationary point properties.
\zcref{sec:merit:relation} then discusses the connection between different merit functions.
In addition, \zcref{sec:merit:lb} gives sufficient conditions under which the merit functions are levfel-bounded.
Finally, \zcref{sec:merit:eb} extends the proximal-PL condition mentioned in \zcref{sec:preliminaries:PL} to multi-objective optimization and shows that the merit functions provide error bounds under such a condition.
\end{document}
