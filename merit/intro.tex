\documentclass[../main]{subfiles}

\begin{document}
\section{Introduction}
This chapter considers the convex-constrained multi-objective optimization problems, i.e.,~\cref{eq:MOO} with~$C$ being non-empty, closed, and convex.
It presents new merit functions for them, and discusses their properties mentioned in \cref{sec:intro:merit}.

In detail, it proposes the following three merit functions for~\cref{eq:MOO}: 
\begin{enumerate}
    \item the gap function for continuous multi-objective optimization; \label{enum:gap_MO}
    \item the regularized gap function for convex multi-objective optimization; \label{enum:reg_gap_MO}
    \item the regularized and partially linearized gap function for composite multi-objective optimization. \label{enum:reg_lin_gap_MO}
\end{enumerate}
In \cref{tab:merit}, we summarize the properties of those merit functions, which will be shown in the subsequent sections.
There, `Sol.' represents the types of Pareto solutions for~\cref{eq:MOO} corresponding to the minima (zero points) of the merit functions.
Moreover, `SP,' `LB,' and `EB' indicate each~$F_i$'s sufficient conditions so that stationary points of the merit functions can solve~\cref{eq:MOO}, the merit functions are level-bounded, and the merit functions provide error bounds, respectively.
The gap function~\cref{enum:gap_MO} connects its minima and the weak Pareto solutions of~\cref{eq:MOO} but does not have good properties in other aspects.
The regularized gap function~\cref{enum:reg_gap_MO} has better properties but requires the convexity of~$F_i$.
The regularized and partially linearized gap function~\cref{enum:reg_lin_gap_MO} relaxes the convexity assumption and is easy to compute for particular problems.
\begin{table}[htpb]
    \caption{Properties of our proposed merit functions}
    \label{tab:merit}
    \begin{subtable}{\textwidth}
        \centering
        \caption{Proposed merit functions and their properties}
        \begin{tblr}{hline{1,2,5}={solid}, hline{3,4}={dashed}, width=\textwidth, colspec={@{}X[.75,c]X[1.5,c]X[c]X[c]X[c]X[1.5,c]X[3,c]X[2,c]@{}}, rowspec={Q[m]Q[m]Q[m]Q[m]}}
        & Obj. & Sol. & Cont. & Diff. & SP & LB & EB \\
            \cref{enum:gap_MO} & Cont. & \SetCell[r=2]{m}WPO & LSC & $\times$ & $\times$ & LB &\SetCell[r=2]{m} SgC \\
            \cref{enum:reg_gap_MO} & Conv. & & \SetCell[r=2]{m}Cont. &\SetCell[r=2]{m} DD & SC & Conv., LB & \\
            \cref{enum:reg_lin_gap_MO} & Comp. & PS & & & SC, $C^2$ & Conv., LB, etc. & SgC, etc.  \\
        \end{tblr}
    \end{subtable}

    \bigskip
    \begin{subtable}{\textwidth}
        \centering
        \caption{Table of abbreviations}
        \begin{tblr}{@{}cc@{}}
            \hline
            Obj. & Objective functions \\
            Sol. & Solutions \\
            Cont. & Continuity \\
            Diff. & Differentiability \\
            SP & Statioary points \\
            LB & Level-boundedness \\
            EB & Error bounds \\
            Cont. & Continuity \\
            Comp. & Composite \\
            WPO & Weak Pareto optimality \\
            PS & Pareto stationarity \\
            LSC & Lower semicontinuity \\
            DD & Directional differentiability \\
            $C^2$ & Twice continuously differentiable \\
            SC & Strict convexity \\
            SgC & Strong convexity \\
            \hline
        \end{tblr}
    \end{subtable}
\end{table}

We summarize the structure of the rest of this chapter.
\Cref{sec:merit:merit} proposes different merit functions for multi-objective optimization with continuous objectives, convex objectives, and composite objectives, along with methods for evaluating the function values, the differentiability, and the stationary point properties.
Furthermore, sufficient conditions for them to be level-bounded and to provide error bounds are given in \cref{sec:merit:lb,sec:merit:eb}, respectively.

\end{document}
