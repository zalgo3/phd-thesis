\documentclass[../main]{subfiles}

\begin{document}
\section{Relation between different merit functions} \zlabel{sec:merit:relation}
This section assumes that the problem has a composite structure~\zcref{eq:composite_MO} and discusses the connection between the merit functions proposed in \zcref{sec:merit:merit}.
First, we show some inequalities between different types of merit functions.
\begin{theorem} \zlabel{thm:merit_between}
    Let~$u_0$, $u_\alpha$, and $w_\alpha$ be defined by~\zcref{eq:gap_MO,eq:reg_gap_MO,eq:reg_lin_gap_MO}, respectively, for all~$\alpha > 0$.
    Then, the following statements hold.
    \begin{enumerate}
        \item If~$f_i$ is~$\mu_i$-convex for some~$\mu_i \in \setR$ and~$\mu = \min_{i = 1, \dots, m} \mu_i$, then we have
            \begin{equation}
                \begin{dcases}
                    u_0(x) \le w_\mu(x) \eqand u_\alpha(x) \le w_{\mu + \alpha}(x), & \text{if } \mu \ge 0, \\
                    u_{- \mu + \alpha}(x) \le w_\alpha(x), & \text{otherwise}
                \end{dcases}
            \end{equation}
            for all~$\alpha > 0$ and~$x \in C$. \zlabel{thm:merit_between:convex}

        \item If~$\nabla f_i$ is $L_{f_i}$-Lipschitz continuous for some~$L_{f_i} > 0$ and~$L_f \coloneqq \max_{i = 1, \dots, m} L_{f_i}$, then we get
            \begin{equation}
                u_{L_f + \alpha}(x) \le w_\alpha(x), \quad u_0(x) \ge w_L(x), \eqand u_\alpha(x) \ge w_{L_f + \alpha}(x)
            \end{equation}
            for all~$\alpha > 0$ and~$x \in C$.
            \zlabel{thm:merit_between:Lipschitz}
    \end{enumerate}
\end{theorem}
\begin{proof}
    \zcref[S]{thm:merit_between:convex}:
Let~$i = 1, \dots, m$.
The $\mu_i$-convexity of~$f_i$ gives
\begin{equation}
    f_i(x) - f_i(y) \le \innerp{\nabla f_i(x)}{x - y} - \frac{\mu_i}{2} \norm{x - y}_2^2
.\end{equation}
By the definition of~$\mu$, we get
\begin{equation}
    f_i(x) - f_i(y) \le \innerp{\nabla f_i(x)}{x - y} - \frac{\mu}{2} \norm{x - y}_2^2
.\end{equation}
Thus, recalling~\zcref{eq:composite_MO}, we have
\begin{align}
    &F_i(x) - F_i(y) \le \innerp{\nabla f_i(x)}{x - y} + g_i(x) - g_i(y) - \frac{\mu}{2} \norm{x - y}^2, \\
    &F_i(x) - F_i(y) - \frac{\alpha}{2}\norm{x - y}_2^2 \le \innerp{\nabla f_i(x)}{x - y} + g_i(x) - g_i(y) - \frac{\mu + \alpha}{2} \norm{x - y}^2, \\
    &F_i(x) - F_i(y) - \frac{- \mu + \alpha}{2}\norm{x - y}_2^2 \le \innerp{\nabla f_i(x)}{x - y} + g_i(x) - g_i(y) - \frac{\alpha}{2} \norm{x - y}^2
,\end{align}
so the desired inequalities are clear from~\zcref{eq:gap_MO,eq:reg_gap_MO,eq:reg_lin_gap_MO}.

\zcref[S]{thm:merit_between:Lipschitz}:
Let~$i = 1, \dots, m$.
Suppose that~$\nabla f_i$ is $L_{f_i}$-Lipschitz continuous.
Then, \zcref{thm:descent} yields
\begin{equation}
    \abs*{f_i(y) - f_i(x) - \innerp{\nabla f_i(x)}{y - x}} \le \frac{L_{f_i}}{2}\norm{x - y}_2^2.
\end{equation}
By the definition of~$L$, we have
\begin{equation}
    \abs*{f_i(y) - f_i(x) - \innerp{\nabla f_i(x)}{y - x}} \le \frac{L_f}{2}\norm{x - y}_2^2.
\end{equation}
This gives
\begin{align}
    &F_i(x) - F_i(y) - \frac{L_f + \alpha}{2} \norm{x - y}_2^2 \le \innerp{\nabla f_i(x)}{x - y} + g_i(x) - g_i(y) - \frac{\alpha}{2} \norm{x - y}_2^2, \\
    &F_i(x) - F_i(y) \ge \innerp{\nabla f_i(x)}{x - y} + g_i(x) - g_i(y) - \frac{L_f}{2} \norm{x - y}_2^2, \\
    &F_i(x) - F_i(y) - \frac{\alpha}{2} \norm{x - y}_2^2 \ge \innerp{\nabla f_i(x)}{x - y} + g_i(x) - g_i(y) - \frac{L_f + \alpha}{2} \norm{x - y}_2^2
.\end{align}
Therefore, we immediately get~$u_{L_f + \alpha}(x) \le w_\alpha(x)$,~$u_0(x) \ge w_L(x)$, and~$u_\alpha(x) \ge w_{L_f + \alpha}(x)$ for all~$x \in C$ by~\zcref{eq:gap_MO,eq:reg_gap_MO,eq:reg_lin_gap_MO}.
\end{proof}

Second, we present the relation between coefficients and the proposed merit functions' values.
\begin{theorem} \zlabel{thm:merit_inner}
    Recall that~$w_{\alpha_1}$ is defined by~\zcref{eq:reg_lin_gap_MO} for all~$\alpha_1 > 0$.
    Let~$\alpha_2$ be an arbitrary scalar such that $\alpha_2 \ge \alpha_1$.
    Then, we get
    \begin{equation}
        w_{\alpha_2}(x) \le w_{\alpha_1}(x) \le \frac{\alpha_2}{\alpha_1} w_{\alpha_2}(x) \forallcondition{x \in C}
    .\end{equation}
\end{theorem}
\begin{proof}
    Let~$x \in C$.
    Since~$\alpha_2 \ge \alpha_1 > 0$, the definition~\zcref{eq:reg_lin_gap_MO} of~$w_{\alpha_2}$ and~$w_{\alpha_1}$ clearly gives the first inequality.
    Thus, we prove the second one.
    From the definition~\zcref{eq:reg_lin_gap_MO} of $w_{\alpha_1}$, we have
\begin{align}
    \MoveEqLeft w_{\alpha_1}(x) = \sup_{y \in C} \min_{i = 1, \dots, m} \left[ \innerp{\nabla f_i(x)}{x - y} + g_i(x) - g_i(y) - \frac{\alpha_1}{2}\norm{x - y}_2^2 \right] \\
    ={}& \frac{\alpha_2}{\alpha_1} \sup_{y \in C} \min_{i = 1, \dots, m} \left[ \innerp*{\nabla f_i(x)}{ \frac{\alpha_1}{\alpha_2}(x - y) } + \frac{\alpha_1}{\alpha_2}(g_i(x) - g_i(y)) - \frac{\alpha_2}{2} \norm*{ \frac{\alpha_1}{\alpha_2} (x - y) }_2^2 \right] \\
    \le{}& \frac{\alpha_2}{\alpha_1} \sup_{y \in C} \min_{i = 1, \dots, m} \left[ \innerp*{\nabla f_i(x)}{ \frac{\alpha_1}{\alpha_2} (x - y) } + g_i(x) - g_i \left(x - \frac{\alpha_1}{\alpha_2}(x - y)\right) \right. \\
        &\hspace{7em} \left. - \frac{\alpha_2}{2}\norm*{\frac{\alpha_1}{\alpha_2}(x - y)}_2^2 \right]
\end{align}
where the first inequality follows from the convexity of~$g_i$.
Since~$C$ is convex, $x, y \in C$ implies~$x - (\alpha_1 / \alpha_2)(x - y) \in C$.
Therefore, from the definition~\zcref{eq:reg_lin_gap_MO} of~$w_{\alpha_2}$, we get
\begin{equation}
    w_{\alpha_1}(x) \le \frac{\alpha_2}{\alpha_1} w_{\alpha_2}(x)
.\end{equation} 
\end{proof}

Considering \zcref{rem:reg_lin_gap:reg_gap}, we get the following corollary.
\begin{corollary}
    Assume that each component~$F_i$ of the objective function~$F$ of~\zcref{eq:MOO} is convex.
    Recall that~$u_{\alpha_1}$ is defined by~\zcref{eq:reg_gap_MO} for all~$\alpha_1 > 0$.
    Let~$r$ be an arbitrary scalar such that $\alpha_2 \ge \alpha_1$.
    Then, we get
    \begin{equation}
        u_{\alpha_2}(x) \le u_{\alpha_1}(x) \le \frac{\alpha_2}{\alpha_1} u_r(x) \forallcondition{x \in C}
    .\end{equation}
\end{corollary}
\end{document}
