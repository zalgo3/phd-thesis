\documentclass[../main]{subfiles}

\begin{document}
\section{Level-boundedness of the proposed merit functions} \label{sec:merit:lb}
Recall that we call a function \emph{level-bounded} if every level set is bounded.
This is an important property because it ensures that the sequences generated by descent methods have accumulation points.
We state below sufficient conditions for the level-boundedness of the merit functions proposed in \cref{sec:merit}.
\begin{theorem}
    Let~$u_0$,~$u_\alpha$ and~$w_\alpha$ be defined by~\cref{eq:u_0,eq:u_alpha,eq:w_alpha}, respectively, for all~$\alpha > 0$.
    Then, the following claims hold.
    \begin{enumerate}
        \item If~$F_i$ is level-bounded for all~$i = 1, \dots, m$, then~$u_0$ is level-bounded. \label{enum:u_0 level-bounded}
        \item If~$F_i$ is convex and level-bounded for all~$i = 1, \dots, m$, then~$u_\alpha$ is level-bounded for all~$\alpha > 0$. \label{enum:u_alpha level-bounded}
        \item Suppose that~$F$ has the composite structure~\cref{eq:composite_MO}.
            If~$f_i$ is~$\mu_i$-convex for some~$\mu_i \in \setR$ or~$\nabla f_i$ is $L_i$-Lipschitz continuous for some~$L_i > 0$, and~$F_i$ is convex and level-bounded for all~$i = 1, \dots, m$, then~$w_\alpha$ is level-bounded for all~$\alpha > 0$. \label{enum:w level-bounded}
    \end{enumerate}
\end{theorem}
\begin{proof}
    \subCref{enum:u_0 level-bounded}: Suppose, contrary to our claim, that~$u_0$ is not level-bounded.
    Then, there exists~$\alpha \in \setR$ such that~$\Set*{ x \in C }{ u_0(x) \le \alpha }$ is unbounded.
    By the definition~\cref{eq:u_alpha} of~$u_0$, the inequality~$u_0(x) \le \alpha$ can be written as
    \[
        \sup_{y \in C} \min_{i = 1, \dots, m} [ F_i(x) - F_i(y) ] \le \alpha.
    \]
    This implies that for some fixed~$z \in C$, there exists~$j = 1, \dots, m$ such that
    \[
        F_j(x) \le F_j(z) + \alpha.
    \]
    Therefore, it follows that
    \[
        \Set{ x \in C}{ u_0(x) \le \alpha } \subseteq \bigcup_{j = 1}^m \Set{ x \in C }{ F_j(x) \le F_j(z) + \alpha }.
    \]
    Since $F_i$ is level-bounded for all $i = 1, \dots, m$, the right-hand side must be bounded, which contradicts the unboundedness of the left-hand side.

    \subCref{enum:u_alpha level-bounded}: Recall the definitions~\cref{eq:simplex,eq:Moreau_env,eq:prox} of~$\Delta^m,\envelope$, and~$\prox$.
    \Cref{eq:u dual} gives
    \[
        \begin{split}
            u_\alpha(x) &= \min_{\lambda \in \Delta^m} \left[ \sum_{i = 1}^{m} \lambda_i F_i(x) - \alpha \envelope_{\frac{1}{\alpha} \sum_{i = 1}^{m} \lambda_i F_i + \indicator_S}(x) \right] \\
                      &\begin{multlined}
                          = \min_{\lambda \in \Delta^m} \sum_{i = 1}^{m} \lambda_i \left[ F_i(x) - F_i\left( \prox_{\frac{1}{\alpha} \sum_{i = 1}^{m} \lambda_i F_i + \indicator_S}(x) \right) \right. \\
                          \left.- \frac{\alpha}{2} \norm*{x - \prox_{\frac{1}{\alpha} \sum_{i = 1}^{m} \lambda_i F_i + \indicator_S}(x)}_2^2 \right] 
                      \end{multlined} \\
                      &\ge \frac{1}{2} \min_{\lambda \in \Delta^m} \sum_{i = 1}^{m} \lambda_i \left[ F_i(x) - F_i\left( \prox_{\frac{1}{\alpha} \sum_{i = 1}^{m} \lambda_i F_i + \indicator_S}(x) \right) \right] \\
                      &= \frac{1}{2} \min_{i = 1, \dots, m} \left[ F_i(x) - F_i\left( \prox_{\frac{1}{\alpha} \sum_{i = 1}^{m} \lambda_i F_i + \indicator_S}(x) \right) \right] 
        ,\end{split}
    \] 
    where the inequality follows from \cref{thm:second_prox}.
    Therefore, with similar arguments given in the proof of \subcref{enum:u_0 level-bounded}, we can show the level-boundedness of~$u_\alpha$ by contradiction.

    \subCref{enum:w level-bounded}: From \cref{thm:merit between,thm:merit inner}, there exist some~$a > 0$ and~$r > 0$ such that~$u_r(x) \le a w_\alpha(x)$ for all~$x \in C$.
    Since \subcref{enum:u_alpha level-bounded} implies that~$u_r$ is level-bounded,~$w_\alpha$ is also level-bounded.
\end{proof}
As indicated by the following example, our proposed merit functions are not necessarily level-bounded even if $F$ is level-bounded.
\begin{example}
    Consider the bi-objective function $F \colon \setR \to \setR^2$ with each component given by
    \[
        F_1(x) \coloneqq x^2, \quad F_2(x) \coloneqq 0.
    \]
    Then, the merit function~$u_0$ defined by~\cref{eq:u_alpha} is written as
    \begin{align}
        u_0(x) & = \sup_{y \in \setR} \min [ F_1(x) - F_1(y), F_2(x) - F_2(y) ] \\
             & = \sup_{y \in \setR} \min [ (x^2 - y^2), 0 ] = 0.
    \end{align}
    On the other hand, $F$ is level-bounded because $\lim_{\norm{x}_2 \to \infty} F_1(x) = \infty$.
\end{example}

\end{document}
