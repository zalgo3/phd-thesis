\documentclass[../main]{subfiles}

\begin{document}
\section{Level-boundedness of the proposed merit functions} \zlabel{sec:merit:lb}
As we discussed in~\zcref{sec:preliminaries:conv}, we call a function \emph{level-bounded} if every level set is bounded.
This is an essential property because it ensures that the sequences generated by descent methods have accumulation points.
We state below sufficient conditions for the level-boundedness of the merit functions proposed in \zcref{sec:merit:merit}.
\begin{theorem} \zlabel{thm:merit_lb}
    Let~$u_\infty$,~$u_\alpha$, and~$w_\alpha$ be defined by~\zcref{eq:gap_MO,eq:reg_gap_MO,eq:reg_lin_gap_MO}, respectively, for all~$\alpha > 0$.
    Then, the following claims hold.
    \begin{enumerate}
        \item If~$F_i$ is level-bounded for all~$i = 1, \dots, m$, then~$u_\infty$ is level-bounded. \zlabel{thm:merit_lb:gap}
        \item If~$F_i$ is convex and level-bounded for all~$i = 1, \dots, m$, then~$u_\alpha$ is level-bounded for all~$\alpha > 0$. \zlabel{thm:merit_lb:reg_gap}
        \item Suppose that~$F$ has the composite structure~\zcref{eq:composite_MO}.
            If~$f_i$ is~$\mu_{f_i}$-convex for some~$\mu_{f_i} \in \setR$ or~$\nabla f_i$ is $L_{f_i}$-Lipschitz continuous for some~$L_{f_i} > 0$, and~$F_i$ is convex and level-bounded for all~$i = 1, \dots, m$, then~$w_\alpha$ is level-bounded for all~$\alpha > 0$. \zlabel{thm:merit_lb:reg_lin_gap}
    \end{enumerate}
\end{theorem}
\begin{proof}
    \zcref[S]{thm:merit_lb:gap}: Suppose, contrary to our claim, that~$u_\infty$ is not level-bounded.
    Then, there exists~$c \in \setR$ such that~$\Set*{ x \in C }{ u_\infty(x) \le c }$ is unbounded.
    By the definition~\zcref{eq:reg_gap_MO} of~$u_\infty$, the inequality~$u_\infty(x) \le c$ can be written as
    \begin{equation}
        \sup_{y \in C} \min_{i = 1, \dots, m} [ F_i(x) - F_i(y) ] \le c.
    \end{equation}
    This implies that for some fixed~$z \in C$, there exists~$j = 1, \dots, m$ such that
    \begin{equation}
        F_j(x) \le F_j(z) + c.
    \end{equation}
    Therefore, it follows that
    \begin{equation}
        \Set{ x \in C}{ u_\infty(x) \le c } \subseteq \bigcup_{j = 1}^m \Set{ x \in C }{ F_j(x) \le F_j(z) + c }.
    \end{equation}
    Since $F_i$ is level-bounded for all $i = 1, \dots, m$, the right-hand side must be bounded, which contradicts the unboundedness of the left-hand side.

    \zcref[S]{thm:merit_lb:reg_gap}: Recall the definitions~\zcref{eq:simplex,eq:Moreau_env,eq:prox} of~$\Delta^m,\envelope$, and~$\prox$.
    \zcref[S]{eq:u dual} gives
    \begin{equation}
        \begin{aligned}
            u_\alpha(x) &= \min_{\lambda \in \Delta^m} \left[ \sum_{i = 1}^{m} \lambda_i F_i(x) - \alpha^{-1} \envelope_{\alpha \sum\limits_{i = 1}^{m} \lambda_i F_i + \indicator_C}(x) \right] \\
                      &\begin{multlined}
                          = \min_{\lambda \in \Delta^m} \sum_{i = 1}^{m} \lambda_i \left[ F_i(x) - F_i\left( \prox_{\alpha \sum\limits_{i = 1}^{m} \lambda_i F_i + \indicator_C}(x) \right) \right. \\
                          \left.- \frac{1}{2 \alpha} \norm*{x - \prox_{\alpha \sum\limits_{i = 1}^{m} \lambda_i F_i + \indicator_C}(x)}_2^2 \right] 
                      \end{multlined} \\
                      &\ge \frac{1}{2} \min_{\lambda \in \Delta^m} \sum_{i = 1}^{m} \lambda_i \left[ F_i(x) - F_i\left( \prox_{\alpha \sum\limits_{i = 1}^{m} \lambda_i F_i + \indicator_C}(x) \right) \right] \\
                      &= \frac{1}{2} \min_{i = 1, \dots, m} \left[ F_i(x) - F_i\left( \prox_{\alpha \sum\limits_{i = 1}^{m} \lambda_i F_i + \indicator_C}(x) \right) \right] 
        ,\end{aligned}
    \end{equation} 
    where the inequality follows from \zcref{thm:second_prox_col}.
    Therefore, with similar arguments given in the proof of \zcref{thm:merit_lb:gap}, we can show the level-boundedness of~$u_\alpha$ by contradiction.

    \zcref[S]{thm:merit_lb:reg_lin_gap}: From \zcref{thm:merit_between,thm:merit_inner}, there exist some~$\tau > 0$ and~$\beta > 0$ such that~$u_\beta(x) \le \tau w_\alpha(x)$ for all~$x \in C$.
    Since \zcref{thm:merit_lb:reg_gap} implies that~$u_\beta$ is level-bounded,~$w_\alpha$ is also level-bounded.
\end{proof}
The following example indicates that our proposed merit functions are not necessarily level-bounded, even if $F$ is level-bounded.
\begin{example}
    Consider the bi-objective function $F \colon \setR \to \setR^2$ with each component given by
    \begin{equation}
        F_1(x) \coloneqq x^2, \quad F_2(x) \coloneqq 0.
    \end{equation}
    Then, the gap function~$u_\infty$ defined by~\zcref{eq:reg_gap_MO} is written as
    \begin{align}
        u_\infty(x) & = \sup_{y \in \setR} \min [ F_1(x) - F_1(y), F_2(x) - F_2(y) ] \\
             & = \sup_{y \in \setR} \min [ (x^2 - y^2), 0 ] = 0.
    \end{align}
    On the other hand, $F$ is level-bounded because $\lim_{\norm{x}_2 \to \infty} F_1(x) = \infty$.
\end{example}

\end{document}
