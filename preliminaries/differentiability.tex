\documentclass[../main]{subfiles}

\begin{document}
\section{Differentiability}
Suppose that~$h \colon \setR^p \to (-\infty, +\infty]$ is finite-valued in an appropriate neighborhood of~$x \in \setR^p$.
    If~$h$ has the partial derivative
    \begin{equation} \label{eq:pdv}
            \pdv{h(x)}{x_i} \coloneqq \lim_{\alpha \to 0} \frac{h\left(x + \alpha e^i\right) - h(x)}{\alpha} \forallcondition{i = 1, \dots, p}
        \end{equation}
    with $e^i$ being the unit vector along the~$x_i$-axis, and if
    \begin{equation} \label{eq:first_order_approximation}
            h(x + \varepsilon) = h(x) + \innerp{\nabla h(x)}{\varepsilon} + o(\norm{\varepsilon}_2) \forallcondition{\varepsilon \in \setR^p}
        \end{equation}
    with~$o \colon [0, +\infty) \to \setR$ satisfying~$\lim_{t \to 0} o(t) / t = 0$ and
\begin{equation} \label{eq:grad}
    \nabla h(x) \coloneqq
    \begin{bmatrix}
        \pdv{h(x)}{x_1} \\ \vdots \\ \pdv{h(x)}{x_p}
    \end{bmatrix}
    ,\end{equation}
then~$h$ is \emph{differentiable} at~$x$, and we call~$\nabla h(x) \in \setR^p$ a \emph{gradient} of~$h$ at~$x$.
If~$\nabla h(x)$ is continuous at~$x$, we say that~$h$ is \emph{continuously differentiable} at~$x$.
Again, if~$h$ has second-order derivatives and
\begin{equation}
    h(x + \varepsilon) = h(x) + \innerp{\nabla h(x)}{\varepsilon} + \frac{1}{2} \innerp*{\varepsilon}{\nabla^2 h(x) \varepsilon} + o\left(\norm{h}_2^2\right)
\end{equation}
with
\begin{equation} \label{eq:hess}
    \nabla^2 h(x) \coloneqq
    \begin{bmatrix}
        \pdv{h(x)}{x_1,x_1} & \cdots & \pdv{h(x)}{x_1,x_p} \\
        \vdots              & \ddots & \vdots              \\
        \pdv{h(x)}{x_p,x_1} & \cdots & \pdv{h(x)}{x_p,x_p} \\
    \end{bmatrix}
    ,\end{equation}
then~$h$ is \emph{twice differentiable} at~$x \in \setR^p$, and~$\nabla^2 h(x)$ is the \emph{Hessian matrix} of~$h$ at~$x$.
When~$\nabla^2 h$ is continuous at~$x$,~$h$ is \emph{twice continuously differentiable} at~$x$, and then~$\nabla^2 h(x)$ is symmetric.
On the other hand, for a vector-valued function~$h \colon \setR^p \to \setR^q$ with~$h \coloneqq (h_1, \dots, h_m)^\T$,~$\jdv{h}{x}$ denotes the Jacobian matrix of~$h$ at~$x$, that is,
\begin{equation} \label{eq:jac}
    \jdv{h}{x} \coloneqq
    \begin{bmatrix}
        \pdv{h_1(x)}{x_1} & \cdots & \pdv{h_1(x)}{x_p} \\
        \vdots            & \ddots & \vdots            \\
        \pdv{h_q(x)}{x_1} & \cdots & \pdv{h_q(x)}{x_p}
    \end{bmatrix}
    = [ \nabla h_1(x), \dots, \nabla h_q(x) ]^\T \in \setR^{q \times p}
    ,\end{equation}
where~$\T$ denotes transpose.

\end{document}
