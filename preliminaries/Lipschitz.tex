\documentclass[../main]{subfiles}

\begin{document}
\section{H\"older and Lipschitz continuity}
We call~$h \colon \setR^p \to \setR$ to be \emph{locally H\"older continuous} with exponent~$\beta > 0$ if for every bounded set~$\Omega \subseteq \setR^p$ there exists~$L_h > 0$ such that
\begin{equation}
    \abs{h(x) - h(y)} \le L_h \norm{x - y}_2^\beta \forallcondition{x, y \in \Omega}
.\end{equation} 
In particular, when~$L_h$ does not depend on~$\Omega$, we say that~$h$ is H\"older continuous with exponent~$\beta > 0$.
Moreover, we refer to the (local) H\"older continuity with exponent~$1$ as the \emph{(local) Lipschitz continuity}.
When~$h$ is Lipschitz continuous, we call~$L_h$ the \emph{Lipschitz constant}, and we also say that~$h$ is \emph{$L_h$-Lipschitz continuous}.
As the following lemma shows, many functions with \emph{good} properties are locally Lipschitz continuous.
\begin{lemma} \zlabel{thm:local_Lipschitz}
    Continuously differentiable functions and finite-valued convex functions are locally Lipschitz continuous.
\end{lemma}
\begin{proof}
    The former is due to the mean value theorem, and the latter is from~\cite{WayneStateUniversity1972}.
\end{proof}

Furthermore, if~$h$ is continuously differentiable and~$\nabla h$ is $L_h$-Lipschitz continuous, we say that~$h$ is~$L_h$-smooth.
We now recall the so-called descent lemma~\cite[Proposition A.24]{Bertsekas1999} as follows:
\begin{lemma}[Descent Lemma~{\cite[Proposition A.24]{Bertsekas1999}}] \zlabel{thm:descent}
    Let~$h \colon \setR^p \to \setR$ is~$L_h$-smooth on~$\setR^p$ with~$L_h > 0$.
    Then, we have
    \begin{equation}
        \abs{h(y) - h(x) - \innerp{\nabla h(x)}{y - x}} \le \frac{L_h}{2} \norm{x - y}_2^2 \forallcondition{x, y \in \setR^p}
    .\end{equation} 
\end{lemma}

\end{document}
