\documentclass[../main]{subfiles}

\begin{document}
\section{Directional derivatives and subgradients}
A function~$h \colon \setR^p \to (-\infty, +\infty]$ is \emph{directionally differentiable} at~$x \in \setR^p$ in a direction~$d \in \setR^p$ if
\begin{equation} \label{eq:dir_deriv}
    h'(x; d) \coloneqq \lim_{\alpha \searrow 0} \frac{h(x + \alpha d) - h(x)}{\alpha}
\end{equation} 
exists, and then we call~$h'(x; d)$ the \emph{directional derivative} at~$x$ in a direction~$d$.
When~$h$ is differentiable at~$x$, we have~$h'(x; d) = \innerp{\nabla h(x)}{d}$ for all~$d \in \setR^p$.
As the following lemma implies, convex functions are directionally differentiable if we allow~$\pm \infty$ as a limit.
\begin{lemma}[{\cite[Section~4.1]{Bertsekas2003}}] \zlabel{thm:nondecreasing}
    Let~$h \colon \setR^p \to (-\infty, +\infty]$ be convex.
    Then, the function~$h_{x, d} \colon (0, +\infty) \to (-\infty, +\infty]$ defined by
    \begin{equation}
        h_{x, d}(\alpha) \coloneqq \frac{h(x + \alpha d) - h(x)}{\alpha}
    \end{equation} 
    is non-decreasing.
    In particular, it follows that
    \begin{equation}
        h'(x; d) \le h_{x, d}(\alpha) \le h(x + d) - h(x) \forallcondition{x, d \in \setR^p, \alpha \in (0, 1]}
    .\end{equation} 
\end{lemma}
On the other hand, for a proper and convex function~$h \colon \setR^p \to (-\infty, +\infty]$, we call~$\xi \in \setR^p$ a \emph{subgradient} of~$h$ at~$x \in \setR^p$ if
\begin{equation} \label{eq:subgrad}
    h(y) - h(x) \ge \innerp{\xi}{y - x} \forallcondition{y \in \setR^p}
,\end{equation} 
and we write~$\partial h(x)$ the \emph{subdifferential} of~$h$ at~$x$, i.e., the set of all subgradients of~$h$ at~$x$.
When~$h$ is differentiable at~$x$, $\partial h(x)$ amounts to a singular $\set{\nabla h(x)}$.
\end{document}
