\documentclass[../main]{subfiles}

\begin{document}
\section{Pareto optimality}
Let us introduce the concept of optimality for the multi-objective optimization problem~\cref{eq:MOO}.
\begin{definition}[Pareto optimality and weak Pareto optimality] \label{def:Pareto}
    For~\cref{eq:MOO}, we say that $x \in C$ is
    \begin{enumerate}
        \item \emph{Pareto optimal} if there is no~$y \in C$ such that~$F(y) \le F(x)$ and~$F(y) \neq F(x)$; \label{def:Pareto:Pareto}
        \item \emph{weakly Pareto optimal} if there does not exist~$y \in C$ such that~$F(y) < F(x)$. \label{def:Pareto:weak}
    \end{enumerate}
\end{definition}
By definition, weak Pareto optimality containss Pareto optimality, though both definitions reduce to the usual optimality when~$m = 1$.
On the other hand, if the objective functions are non-convex, it is challenging to find Pareto minima or weak Pareto minima.
In such cases, optimization algorithms basically aim to get Pareto stationary points defined as follows:
\begin{definition}[Pareto stationarity] \label{def:Pareto_stationary}
    Assume that~$F_i$ is directionally differentiable for every~$i = 1, \dots, m$ and~$C$ is non-empty, closed, and convex.
    Then, we call~$x \in C$ \emph{Pareto stationary} if~
    \[
        \max_{i = 1, \dots, m} F_i'(x; y - x) \ge 0 \forallcondition{y \in C}
    .\] 
\end{definition}
We state below the relation among the three concepts given by~\cref{def:Pareto,def:Pareto_stationary}
\begin{lemma} \label{thm:Pareto}
    Suppose that~$F_i$ is directionally differentiable for every~$i = 1, \dots, m$ and~$C$ is non-empty, closed, and convex.
    Then, the following three claims hold.
    \begin{enumerate}
        \item If~$x \in C$ is weakly Pareto optimal for~\cref{eq:MOO}, then~$x$ is Pareto stationary for~\cref{eq:MOO}. \label{thm:Pareto:weak_stationary}
        \item Let every~$F_i, i = 1, \dots, m$ be convex. Then, all Pareto stationary points of~\cref{eq:MOO} are weakly Pareto optimal for~\cref{eq:MOO}. \label{thm:Pareto:convex}
        \item Suppose that~$F_i$ is strictly convex for any~$i = 1, \dots, m$. Then, every Pareto stationary points of~\cref{eq:MOO} is Pareto optimal for~\cref{eq:MOO}. \label{thm:Pareto:strict_convex}
    \end{enumerate}
\end{lemma}
\begin{proof}
    We prove each claim's contraposition.

    \subCref{thm:Pareto:weak_stationary}:
    Assume that~$x \in C$ is not Pareto stationary.
    Then, \cref{def:Pareto_stationary} shows that for some~$y \in C$ we have~$\max_{i = 1, \dots, m} F_i'(x; y - x) < 0$.
    By the definition~\cref{eq:dir_deriv} of the directional derivative, for a sufficiently small scalar~$\alpha > 0$, we obtain
    \[
        \max_{i = 1, \dots, m} [ F_i(x + \alpha (y - x)) - F_i(x) ] < 0 
    ,\] 
    which means that~$x$ is not weakly Pareto optimal from \cref{def:Pareto:weak}.

    \subCref{thm:Pareto:convex}:
    Suppose that~$x \in C$ is not weakly Pareto optimal.
    Then, \cref{def:Pareto:weak} implies that there exists~$y \in C$ such that~$F_i(y) < F_i(x)$ for all~$i = 1, \dots, m$.
    Therefore, the convexity of~$F_i$ and \cref{thm:nondecreasing} give
    \[
        F'(x; y - x) \le F_i(y) - F_i(x) < 0 \forallcondition{i = 1, \dots, m}
    .\] 
    Hence, we get
    \[
        \max_{i = 1, \dots, m} F'(x; y - x) < 0
    ,\] 
    which implies that~$x$ is not Pareto stationary from \cref{def:Pareto_stationary}.

    \subCref{thm:Pareto:strict_convex}:
    Suppose that~$x \in C$ is not Pareto optimal.
    From \cref{def:Pareto:Pareto}, there exists~$y \in C$ such that~$F(y) \le F(x)$ and~$F(y) \neq F(x)$.
    Since~$F_i$ is strictly convex for every~$i = 1, \dots, m$, we have
    \[
        F(x + \alpha(y - x)) < F(x) + \alpha (F(y) - F(x)) \forallcondition{\alpha \in (0, 1)}
    .\] 
    Reducing~$F(x)$ and dividing by~$\alpha$ from both sides lead to
    \[
        \frac{F(x + \alpha(y - x)) - F(x)}{\alpha} < F(y) - F(x) \le 0
    .\] 
    Applying \cref{thm:nondecreasing} to each component yields
    \[
        F_i'(x; y - x) < F_i(y) - F_i(x) \le 0
    ,\] 
    which shows that~$x$ is not Pareto stationary.
\end{proof}
\end{document}
