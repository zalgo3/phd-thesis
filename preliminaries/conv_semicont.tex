\documentclass[../main]{subfiles}

\begin{document}
\section{Convexity and semi-continuity} \zlabel{sec:preliminaries:conv}
We first define the convexity of sets and functions.
A set~$C \subseteq \setR^p$ is \emph{convex} if
\begin{equation}
    (1 - \alpha) v^1 + \alpha v^2 \in C \forallcondition{v^1, v^2 \in C, \alpha \in [0, 1]}
.\end{equation} 
Likewise, a function~$h \colon \setR^p \to (-\infty, +\infty]$ is \emph{convex} if
\begin{equation}
    h((1 - \alpha) x + \alpha y) \le (1 - \alpha) h(x) + \alpha h(y) \forallcondition{x, y \in \dom(h), \alpha \in [0, 1]}
,\end{equation}
\emph{strictly convex} if
\begin{equation}
    h((1 - \alpha) x + \alpha y) < (1 - \alpha) h(x) + \alpha h(y) \forallcondition{x, y \in \dom(h), \alpha \in (0, 1)}
,\end{equation} 
and \emph{$\mu_f$-convex} with~$\mu_f \in \setR$ if
\begin{equation}
    h((1 - \alpha) x + \alpha y) \le (1 - \alpha) h(x) + \alpha h(y) \forallcondition{x, y \in \dom(h), \alpha \in [0, 1]}
,\end{equation} 
where~$\dom(h)$ stands for the \emph{effective domain} of~$h$ given by
\begin{equation}
    \dom(h) \coloneqq \Set{x \in \setR^p}{h(x) < + \infty}
.\end{equation} 
In particular, the \emph{strong convexity} (with modulus~$\mu_f$) denotes the~$\mu_f$-convexity with~$\mu_f > 0$.
We also note that the~$0$-convexity is equivalent to the usual convexity.
Moreover, if~$\dom(h) \neq \emptyset$ for some convex function~$h \colon \setR^p (- \infty, + \infty]$, we say that~$h$ is \emph{proper} and convex.
On the other hand, we call~$h$ to be concave if~$- h$ is convex.
Every definition and argument relating to convex functions also holds for concave functions by appropriately interchanging~$\le$ and~$\ge$,~$+ \infty$ and~$- \infty$,~$\sup$ and~$\inf$, etc.

Let us now introduce the semi-continuity of functions.
For all~$\set*{x^k} \subseteq \setR^p$ converging to~$x \in \setR^p$, a function~$h \colon \setR^p \to (-\infty, +\infty]$ is \emph{upper semi-continuous} at~$x$ if
\begin{equation}
    h(x) \ge \limsup_{k \to \infty} h\left(x^k\right) 
\end{equation} 
and \emph{lower semi-continuous} if
\begin{equation}
    h(x) \le \liminf_{k \to \infty} h\left(x^k\right) 
.\end{equation} 
A necessary and sufficient condition for~$h$ to be lower semi-continuous is that the level set~$\level_c(h)$ given by
\begin{equation} \label{eq:level}
    \level_c(h) \coloneqq \Set{x \in \setR^p}{h(x) \le c}
\end{equation} 
is closed for any~$c \in \setR$.
We refer to lower-semi-continuous, proper, and convex functions as \emph{closed, proper, and convex} functions.
The level sets of convex functions are convex, and the level sets of closed, proper, and convex functions are closed and convex.
Note that if~$\level_c(h)$ is bounded for all~$c \in \setR$, we say that~$h$ is \emph{level-bounded}.
For example, every strongly convex function is level-bounded.
Note also that~\zcref{eq:level} is applicable as a definition of the level set for the vector-valued function~$h \colon \setR^p \to \setR^q$ and~$c \in \setR^q$.
\end{document}
