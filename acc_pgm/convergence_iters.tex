\documentclass[../main]{subfiles}

\begin{document}
\section{Convergence of the iterates} \label{sec:acc_pgm:convergence}
While the last section shows that \cref{alg:acc_pgm_MO} has an~$O(1 / k^2)$ convergence rate like \cref{alg:acc_pgm_MO}, this section proves the following theorem:
\begin{theorem} \label{thm:main convergence}
    Let~$\set*{x^k}$ be generated by \cref{alg:acc_pgm_MO} with~$a > 0$.
    Then, under \cref{asm:bound}, the following two properties hold:
    \begin{enumerate}
        \item \label{thm:main convergence:bound} $\set*{x^k}$ is bounded, and it has an accumulation point;
        \item \label{thm:main convergence:Pareto} $\set*{x^k}$ converges to a weak Pareto optimum for~\cref{eq:MOO}.
    \end{enumerate}
\end{theorem}
The latter claim is also significant in application.
For example, finite-time manifold (active set) identification, which detects the low-dimensional manifold where the optimal solution belongs, essentially requires only the convergence of the generated sequence to a unique point rather than the strong convexity of the objective functions~\cite{Sun2019}.

Again, we will prove \cref{thm:main convergence} after showing some lemmas.
First, we mention the following result, obvious from \cref{asm:bound,thm:main theorem of stepsize section:less than x0}.
\begin{lemma} \label{thm:z exist}
    Let~$\set*{x^k}$ be generated by \cref{alg:acc_pgm_MO} and let~$X^\ast$ be the set of weakly Pareto optimal points.
    Then, for any~$k \ge 0$, there exists~$z \in X^\ast \cap \level_F(F(x^0))$ (see \cref{eq:level} for the definition of~$\level_F$) such that
    \[
        \sigma_k(z) \ge 0 \eqand \norm*{z - x^0}^2 \le R
    ,\] 
    where~$R \ge 0$ is given by~\cref{eq:R}.
\end{lemma}

The following lemma also contributes strongly to the proof of the main theorem.
\begin{lemma} \label{thm:prod}
    Let~$\set*{\gamma_q}$ be defined by \cref{line:gamma} in \cref{alg:acc_pgm_MO}.
    Then, we have
    \[
        \sum_{p = s}^r \prod_{q = s}^p \gamma_q \le 2 ( s - 1) \forallcondition{s, r \ge 1}
    .\] 
\end{lemma}
\begin{proof}
    By using \Cref{thm:t:gamma}, we see that
    \[
        \prod_{q = s}^{p} \gamma_q \le \prod_{q = s}^{p} \frac{q - 1}{q + 1 / 2} 
    .\]
    Let~$\Gamma$ and~$\Beta$ denote the gamma and beta functions defined by
    \[ \label{eq:gamma and beta}
        \Gamma(\alpha) \coloneqq \int_{0}^{\infty} \tau^{\alpha - 1} \exp(-\tau) \odif{\tau} \eqand
        \Beta(\alpha, \beta) \coloneqq \int_{0}^{1} \tau^{\alpha - 1} (1 - \tau)^{\beta - 1} \odif{\tau}
    ,\]
    respectively.
    Applying the well-known properties:
    \[ \label{eq:gamma and beta properties}
        \Gamma(\alpha) = (\alpha - 1)!, \quad \Gamma(\alpha + 1) = \alpha \Gamma(\alpha), \eqand B(\alpha, \beta) = \frac{\Gamma(\alpha) \Gamma(\beta)}{\Gamma(\alpha + \beta)}
    .\]
    we get
    \[
        \prod_{q = s}^{p} \gamma_q \le \frac{\Gamma(p) / \Gamma(s - 1)}{\Gamma(p + 3 / 2) / \Gamma(s + 1 / 2)}
        = \frac{B(p, 3 / 2)}{B(s - 1, 3 / 2)}
    .\] 
    This implies
    \[
        \sum_{p = s}^{r} \prod_{q = s}^{p} \gamma_q \le \sum_{p = 1}^{r} B(p, 3 / 2) / B(s - 1, 3 / 2)
    .\] 
    Then, it follows from the definition~\cref{eq:gamma and beta} of~$\Beta$ that
    \[
        \begin{split}
            \sum_{p = s}^{r} \prod_{q = s}^{p} \gamma_q 
        &\le \sum_{p = s}^{r} \int_{0}^{1} \tau^{p - 1} (1 - \tau)^{1 / 2} \odif{\tau} / B(s - 1, 3 / 2) \\
        &= \int_{0}^{1} \sum_{p = s}^{r} \tau^{p - 1} (1 - \tau)^{1 / 2} \odif{\tau} / B(s - 1, 3 / 2) \\
        &= \int_{0}^{1} \frac{\tau^{s - 1} - \tau^r}{1 - \tau} (1 - \tau)^{1 / 2} \odif{\tau} / B(s - 1, 3 / 2) \\
        &= \frac{B(s, 1 / 2) - B(r + 1, 1 / 2)}{B(s - 1, 3 / 2)} 
        \le \frac{B(s, 1 / 2)}{B(s - 1, 3 / 2)}
        .\end{split}
    \] 
    Using again~\cref{eq:gamma and beta properties}, we conclude that
    \[
        \sum_{p = s}^{r} \prod_{q = s}^{p} \gamma_q
        \le \frac{\Gamma(s) \Gamma(1 / 2) / \Gamma(s + 1 / 2)}{\Gamma(s - 1) \Gamma(3 / 2) / \Gamma(s + 1 / 2)}
        = 2 (s - 1)
    .\] 
\end{proof}

Now, we introduce two functions~$\omega_k \colon \setR^n \to \setR$ and~$\nu_k \colon \setR^n \to \setR$ for any~$k \ge 1$, which will help our analysis, by
\begin{align}
    \omega_k(z) &\coloneqq \max \left( 0, \norm*{x^k - z}^2 - \norm*{x^{k - 1} - z}^2 \right) \label{eq:omega}, \\
    \nu_k(z) &\coloneqq \norm*{x^k - z}^2 - \sum_{s = 1}^{k} \omega_s(z) \label{eq:nu}
.\end{align}
The lemma below describes the properties of~$\omega_k$ and~$\nu_k$.
\begin{lemma} \label{thm:omega nu}
    Let~$\set*{x^k}$ be generated by \cref{alg:acc_pgm_MO} and recall that~$\level_F, \omega_k$, and~$\nu_k$ are defined by~\cref{eq:level,eq:omega,eq:nu}, respectively.
    Moreover, suppose that \cref{asm:bound} holds and that~$z \in X^\ast \cap \level_F(F(x^0))$ satisfies the statement of \cref{thm:z exist} for some~$k \ge 1$.
    Then, it follows for all~$r = 1, \dots, k$ that
    \begin{enumerate}
        \item $\displaystyle \sum_{s = 1}^{r} \omega_s(z) \le \sum_{s = 1}^{r}(6s - 5) \norm*{x^s - x^{s - 1}}^2;$ \label{thm:omega nu:omega}
        \item $\displaystyle \nu_{r + 1}(z) \le \nu_r(z).$ \label{thm:omega nu:nu}
    \end{enumerate}
\end{lemma}
\begin{proof}
    \sublabelcref{thm:omega nu:omega}:
    Let~$k \ge p \ge 1$.
    From the definition of~$y^{p + 1}$ given in \cref{line:y} of \cref{alg:acc_pgm_MO}, we have
    \[
        \begin{split}
            &\norm*{x^{p + 1} - z}^2 - \norm*{x^p - z}^2 \\
        &= - \norm*{x^{p + 1} - x^p}^2 + 2 \innerp*{x^{p + 1} - y^{p + 1}}{x^{p + 1} - z} + 2 \gamma_p \innerp*{x^p - x^{p - 1}}{x^{p + 1} - z} \\
        &= - \norm*{x^{p + 1} - x^p}^2 + 2 \innerp*{x^{p + 1} - y^{p + 1}}{y^{p + 1} - z} + 2 \norm*{x^{p + 1} - y^{p + 1}}^2 \\
        &\quad + 2 \gamma_p \innerp*{x^p - x^{p - 1}}{x^{p + 1} - z}
     .   \end{split}
    \]
    On the other hand, \cref{thm:sigma:1} gives
    \[
        2 \innerp*{x^{p + 1} - y^{p + 1}}{y^{p + 1} - z} \le - \frac{2}{\ell} \sigma_{p + 1}(z) - \frac{2 \ell - L}{\ell} \norm*{x^{p + 1} - y^{p + 1}}^2
    .\]
    Moreover, \cref{thm:sigma k1 k2} with~$(k_1, k_2) = (p + 1, k + 1)$ implies
    \[
        \begin{split}
            \MoveEqLeft
        - \frac{2}{\ell} \sigma_{p + 1}(z) \\
        &\le - \frac{2}{\ell} \sigma_{k + 1}(z) - \norm*{x^{k + 1} - x^k}^2 + \norm*{x^{p + 1} - x^p}^2 - \sum_{r = p + 1}^{k} \frac{1}{t_r}\norm*{x^r - x^{r - 1}}^2 \\
        &\le \norm*{x^{p + 1} - x^p}^2
        ,\end{split}
    \]
    where the second inequality comes from the assumption on~$z$.
    Combining the above three inequalities, we get
    \begin{multline}
        \norm*{x^{p + 1} - z}^2 - \norm*{x^p - z}^2 \le \frac{L}{\ell} \norm*{x^{p + 1} - y^{p + 1}}^2 + 2 \gamma_p \innerp*{x^p - x^{p - 1}}{x^{p + 1} - z} \\
        \shoveleft = \frac{L}{\ell} \norm*{x^{p + 1} - y^{p + 1}}^2
        + \gamma_p \Bigl( \norm*{x^p - z}^2 - \norm*{x^{p - 1} - z}^2 + \norm*{x^p - x^{p - 1}}^2 \\
    + 2 \innerp*{x^p - x^{p - 1}}{x^{p + 1} - x^p} \Bigr)
    .\end{multline}
    Using the relation~$\norm*{x^{p + 1} - y^{p + 1}}^2 + 2 \gamma_p \innerp*{x^p - x^{p - 1}}{x^{p + 1} - x^p} = \norm*{x^{p + 1} - x^p}^2 + \gamma_p^2 \norm*{x^p - x^{p - 1}}^2$, which holds from the definition of~$y^k$, we have
    \begin{multline}
        \norm*{x^{p + 1} - z}^2 - \norm*{x^p - z}^2
        \le - \frac{\ell - L}{\ell} \norm*{x^{p + 1} - y^{p + 1}}^2 + \norm*{x^{p + 1} - x^p}^2 \\
        + \gamma_p \left( \norm*{x^p - z}^2 - \norm*{x^{p - 1} - z}^2 \right) + ( \gamma_p + \gamma_p^2 ) \norm*{x^p - x^{p - 1}}^2
    .\end{multline}
    Since~$0 \le \gamma_p \le 1$ from \cref{thm:t:gamma} and~$\ell \ge L$, we obtain
    \begin{multline}
        \norm*{x^{p + 1} - z}^2 - \norm*{x^p - z}^2 \\
        \le \gamma_p \left( \norm*{x^p - z}^2 - \norm*{x^{p - 1} - z}^2 + 2 \norm*{x^p - x^{p - 1}}^2 \right) + \norm*{x^{p + 1} - x^p}^2 \\
        \le \gamma_p \left( \omega_p(z) + 2 \norm*{x^p - x^{p - 1}}^2 \right) + \norm*{x^{p + 1} - x^p}^2
    ,\end{multline}
    where the second inequality follows from the definition~\cref{eq:omega} of~$\omega_p$.
    Since the right-hand side is nonnegative,~\cref{eq:omega} again gives
    \[
        \omega_{p + 1}(z) \le \gamma_p \left( \omega_p(z) + 2 \norm*{x^p - x^{p - 1}}^2 \right) + \norm*{x^{p + 1} - x^p}^2 
    .\]
    Let~$s \le k$.
    Applying the above inequality recursively and using~$\gamma_1 = 0$, we get
    \[
        \begin{split}
            \omega_s(z) &\le 3 \sum_{p = 2}^{s} \prod_{q = p}^{s} \gamma_q \norm*{x^p - x^{p - 1}}^2 + 2 \prod_{q = 1}^{s} \gamma_q \norm*{x^1 - x^0}^2 + \norm*{x^s - x^{s - 1}}^2 \\
                        &\le 3 \sum_{p = 2}^{s} \prod_{q = p}^{s} \gamma_q \norm*{x^p - x^{p - 1}}^2 + \norm*{x^s - x^{s - 1}}^2
        .\end{split}
    \]
    Adding up the above inequality from~$s = 1$ to~$s = r \le k$, we have
    \[
        \begin{split}
        \sum_{s = 1}^r \omega_s(z)
        &\le 3 \sum_{s = 1}^r \sum_{p = 1}^s \prod_{q = p}^s \gamma_q \norm*{x^p - x^{p - 1}}^2 + \sum_{s = 1}^r \norm*{x^s - x^{s - 1}}^2 \\
        &= 3 \sum_{p = 1}^r \sum_{s = p}^r \prod_{q = p}^s \gamma_q \norm*{x^p - x^{p - 1}}^2 + \sum_{s = 1}^r \norm*{x^s - x^{s - 1}}^2 \\
        &= \sum_{s = 1}^r \left( 3 \sum_{p = s}^r \prod_{q = s}^p \gamma_q + 1 \right) \norm*{x^s - x^{s - 1}}^2
        ,\end{split}
    \]
    where the first equality follows from~\cref{eq:change sum}.
    Thus, \cref{thm:prod} implies
    \[
        \sum_{s = 1}^{r} \omega_s(z) \le \sum_{s = 1}^{r} (6 s - 5) \norm*{x^s - x^{s - 1}}^2 
    .\]

    \sublabelcref{thm:omega nu:nu}:
    \Cref{eq:nu} yields
    \[
        \begin{split}
            \nu_{r + 1}(z) &= \norm*{x^{r + 1} - z}^2 - \omega_{r + 1}(z) - \sum_{s = 1}^r \omega_s(z) \\
                           &= \norm*{x^{r + 1} - z}^2 - \max \left( 0, \norm*{x^{r + 1} - z}^2 - \norm*{x^r - z}^2 \right) - \sum_{s = 1}^{r} \omega_s(z) \\
                           &\le \norm*{x^{r + 1} - z}^2 - \left( \norm*{x^{r + 1} - z}^2 - \norm*{x^r - z}^2 \right) - \sum_{s = 1}^{r} \omega_s(z) \\
                           &= \norm*{x^r - z}^2 - \sum_{s = 1}^{r} \omega_s(z) = \nu_r(z) 
        ,\end{split}
    \] 
    where the second and third equalities come from the definitions~\cref{eq:omega,eq:nu} of~$\omega_{r + 1}$ and~$\nu_r$, respectively.
\end{proof}

Let us now prove the first part of the main theorem.
\begin{proof}[\cref{thm:main convergence:bound}]
    Let~$k \ge 1$ and suppose that~$z \in X^\ast \cap \level_F(F(x^0))$ satisfies the statement of \cref{thm:z exist}, where~$X^\ast$ is the set of weakly Pareto optimal solutions and~$\level_F$ is given by~\cref{eq:level}.
    Then, \cref{thm:omega nu:nu} gives
    \[
        \begin{split}
            \nu_k(z) &\le \nu_1(z) = \norm*{x^1 - z}^2 - \omega_1(z) \\
                           &= \norm*{x^1 - z}^2 - \max \left( 0, \norm*{x^1 - z}^2 - \norm*{x^0 - z}^2 \right) \\
                           &\le \norm*{x^1 - z}^2 - \left( \norm*{x^1 - z}^2 - \norm*{x^0 - z}^2 \right) = \norm*{x^0 - z}^2
        ,\end{split}
    \] 
    where the second equality follows from the definition~\cref{eq:omega} of~$\omega_1$.
    Considering the definition~\cref{eq:nu} of~$\nu_k$, we obtain
    \[
        \norm*{x^k - z}^2 \le \norm*{x^0 - z}^2 + \sum_{s = 1}^{k} \omega_s(z)
    .\] 
    Taking the square root of both sides and using~\cref{eq:omega}, we get
    \[
        \norm*{x^k - z} \le \sqrt{ \norm*{x^0 - z}^2 + \sum_{s = 1}^{k} (6s - 5) \norm*{x^s - x^{s - 1}}^2 }
    .\] 
    Applying the reverse triangle inequality~$\norm*{x^k - x^0} - \norm*{x^0 - z} \le \norm*{x^k - z}$ to the left-hand side leads to
    \begin{align}
        \norm*{x^k - x^0} &\le \norm*{x^0 - z} + \sqrt{ \norm*{x^0 - z}^2 + \sum_{s = 1}^{k} (6s - 5) \norm*{x^s - x^{s - 1}}^2 } \\
                          &\le \sqrt{R} + \sqrt{R + \sum_{s = 1}^{k} (6s - 5) \norm*{x^s - x^{s - 1}}^2}
    ,\end{align}
    where the second inequality comes from the assumption on~$z$.
    Moreover, since~$a > 0$, the right-hand side is bounded from above according to \cref{thm:acc conv rate}.
    This implies that~$\set*{x^k}$ is bounded, and so it has accumulation points.
\end{proof}

Before proving \cref{thm:main convergence:Pareto}, we show the following lemma.
\begin{lemma} \label{thm:convergence norm}
    Let~$\set*{x^k}$ be generated by \cref{alg:acc_pgm_MO} with~$a > 0$ and suppose that \cref{asm:bound} holds.
    Then, if~$\bar{z}$ is an accumulation point of~$\set*{x^k}$, then~$\set*{\norm*{x^k - \bar{z}}}$ is convergent.
\end{lemma}
\begin{proof}
    Assume that~$\set*{x^{k_j}} \subseteq \set*{x^k}$ converges to~$\bar{z}$.
    Then, we have~$\sigma_{k_j}(\bar{z}) \to 0$ by the definition~\cref{eq:sigma rho} of~$\sigma_{k_j}$.
    Therefore, we can regard~$\bar{z}$ to satisfy the statement of \cref{thm:z exist} at~$k = \infty$, and thus the inequalities of \cref{thm:omega nu} hold for any~$r \ge 1$ and~$\bar{z}$.
    This means~$\set*{\nu_k(\bar{z})}$ is non-increasing and bounded, i.e., convergent.
    Hence~$\set*{\norm*{x^k - \bar{z}}}$ is convergent.
\end{proof}

Finally, we finish the proof of the main theorem.
\begin{proof}[\cref{thm:main convergence:Pareto}]
    Suppose that~$\set*{x^{k^1_j}}$ and~$\set*{x^{k^2_j}}$ converges to~$\bar{z}^1$ and~$\bar{z}^2$, respectively.
    From \cref{thm:convergence norm}, we see that
    \[
        \lim_{j \to \infty} \left( \norm*{x^{k^2_j} - \bar{z}^1}^2 - \norm*{x^{k^2_j} - \bar{z}^2}^2 \right) = \lim_{j \to \infty} \left( \norm*{x^{k^1_j} - \bar{z}^1}^2 - \norm*{x^{k^1_j} - \bar{z}^2}^2 \right) 
    .\] 
    This yields that~$\norm*{\bar{z}^1 - \bar{z}^2}^2 = - \norm*{\bar{z}^1 - \bar{z}^2}^2$, and so~$\norm*{\bar{z}^1 - \bar{z}^2}^2 = 0$, i.e.,~$\set*{x^k}$ is convergent.
    Let~$x^k \to x^\ast$.
    Since~$\norm*{x^{k + 1} - x^k}^2 \to 0$, $\set*{y^k}$ is also convergent to~$x^\ast$.
    Therefore, \cref{thm:acc_pgm_stop} shows that~$x^\ast$ is weakly Pareto optimal for~\cref{eq:MOO}.
\end{proof}

\end{document}
