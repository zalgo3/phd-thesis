\documentclass[../main]{subfiles}

\begin{document}
\section{The algorithm} \zlabel{sec:acc_pgm:algorithm}
This section proposes an accelerated version of the proximal gradient method for multi-objective optimization.
Similar to the non-accelerated version given in the last section, a subproblem is considered in each iteration.
More specifically, the proposed method solves the following subproblem for given~$x \in \dom(F)$,~$y \in \setR^n$, and~$\alpha > 0$:
\begin{equation} \label{eq:acc_pgm_subprob}
    \min_{z \in \setR^n} \quad \varphi^\acc_\alpha(z; x, y)
    ,\end{equation}
where
\begin{equation} \label{eq:acc_pgm_obj}
    \varphi^\acc_\alpha(z; x, y) \coloneqq \max_{i = 1, \dots, m} \left[ \innerp{\nabla f_i(y)}{z - y} + g_i(z) + f_i(y) - F_i(x) \right] + \frac{1}{2 \alpha} \norm{z - y}_2^2
    .\end{equation}
Note that when~$y = x$,~\zcref{eq:acc_pgm_subprob} is reduced to the subproblem~\zcref{eq:pgm_subproblem} of the proximal gradient method.
Note also that when~$m = 1$, the subproblem becomes
\begin{equation} \label{eq:single_fista_subprob}
    \min_{z \in \setR^n} \quad \innerp*{\nabla f_1(y)}{z - y} + g_1(z) + \frac{1}{2 \alpha} \norm{z - y}_2^2
    ,\end{equation}
which is the subproblem of the single-objective FISTA (\zcref{alg:acc_pgm}).
The distinctive feature of our proposal~\zcref{eq:acc_pgm_subprob} is the term~$f_i(y) - f_i(x)$, whereas the easy analogy from the single-objective subproblem~\zcref{eq:single_fista_subprob} is
\begin{equation}
    \min_{z \in \setR^n} \quad \max_{i = 1, \dots, m} \left[ \innerp{\nabla f_i(y)}{z - y} + g_i(z) \right] + \frac{1}{2 \alpha} \norm{z - y}_2^2
    .\end{equation}
By putting such a term, the inside of the~$\max$ operator approximates~$F_i(z) - F_i(x)$ rather than~$F_i(z) - F_i(y)$.
This is a negligible difference in the single-objective case but profoundly affects the proof in the multi-objective case.

Since~$g_i$ is convex for all~$i = 1, \dots, m$, $z \mapsto \varphi^\acc_\alpha(z; x, y)$ is strongly convex.
Thus, the subproblem~\zcref{eq:acc_pgm_subprob} has a unique optimal solution~$p^\acc_\alpha(x, y)$ and takes the optimal function value~$\theta^\acc_\alpha(x, y)$, i.e.,
\begin{equation} \label{eq:acc_pgm_optimal_function}
    p^\acc_\alpha(x, y) \coloneqq \argmin_{z \in \setR^n} \varphi^\acc_\alpha(z; x, y) \eqand \theta^\acc_\alpha(x, y) \coloneqq \min_{z \in \setR^n} \varphi^\acc_\alpha(z; x, y)
    .\end{equation}
Moreover, the optimality condition of~\zcref{eq:acc_pgm_subprob} implies
\begin{equation}
    \frac{1}{\alpha} [ y - p^\acc_\alpha(x, y) ] \in \conv_{i \in \mathcal{I}_\alpha(x, y)} [ \nabla f_i(y) + \partial g_i(p^\acc_\alpha(x, y)) ] \forallcondition{x, y \in \setR^n}
\end{equation}
with
\begin{equation} \label{eq:acc_pgm_active}
    \mathcal{I}_\alpha(x, y) \coloneqq \argmax_{i = 1, \dots, m} [ \innerp{\nabla f_i(y)}{p^\acc_\alpha(x, y) - y} + g_i(p^\acc_\alpha(x, y)) + f_i(y) - F_i(x) ]
    .\end{equation}
Hence, for any~$x, y, z \in \setR^n$, there exists~$\lambda(x, y) \in \simplex^m$ such that~$\lambda_j(x, y) = 0$ for~$j \notin \mathcal{I}_\alpha(x, y)$ and
\begin{multline} \label{eq:acc_pgm_opt}
    \frac{1}{\alpha} \innerp{y - p^\acc_\alpha(x, y)}{z - p^\acc_\ell(x, y)} \\
    \le \sum_{i = 1}^{m} \lambda_i(x, y) [\innerp{\nabla f_i(y)}{z - p^\acc_\alpha(x, y)} + g_i(z) - g_i(p^\acc_\alpha(x, y))]
    ,\end{multline}
where~$\simplex^m$ denotes the unit $m$-simplex~\zcref{eq:simplex}.
We also note that by taking~$z = y$ in the objective function of~\zcref{eq:acc_pgm_subprob}, we have
\begin{equation} \label{eq:acc_pgm_opt_triv}
    \theta^\acc_\alpha(x, y) \le \varphi^\acc_\alpha(y; x, y) = \max_{i = 1, \dots, m}\left[ F_i(y) - F_i(x) \right]
\end{equation}
for all~$x \in \dom(F)$ and~$y \in \setR^n$.
We now characterize weak Pareto optimality in terms of the mappings~$p^\acc_\alpha$ and $\theta^\acc_\alpha$, similarly to \zcref{thm:pgm_stop} for the proximal gradient method.
\begin{proposition} \zlabel{thm:acc_pgm_stop}
    Let~$p^\acc_\alpha(x, y)$ and~$\theta^\acc_\alpha(x, y)$ be defined by~\zcref{eq:acc_pgm_optimal_function}.
    Then, the statements below hold.
    \begin{enumerate}
        \item The following three conditions are equivalent:
              \begin{enumerate}
                  \item $y \in \setR^n$ is weakly Pareto optimal for~\zcref{eq:MOO}; \zlabel{thm:acc_pgm_stop:equivalence:Pareto}
                  \item $p^\acc_\alpha(x, y) = y$ for some~$x \in \setR^n$; \zlabel{thm:acc_pgm_stop:equivalence:sol}
                  \item $\theta^\acc_\alpha(x, y) = \max_{i = 1, \dots, m} [ F_i(y) - F_i(x) ]$ for some~$x \in \setR^n$. \zlabel{thm:acc_pgm_stop:equivalence:fun}
              \end{enumerate}\zlabel{thm:acc_pgm_stop:equivalence}
        \item The mappings~$p^\acc_\alpha$ and~$\theta^\acc_\alpha$ are continuous.
              Particularly, if~$\nabla f_i$ is locally Lipschitz continuous,~$p^\acc_\alpha$ and~$\theta^\acc_\alpha$ are locally H\"older continuous with exponent~$1 / 2$ and locally Lipschitz continuous, respectively. \zlabel{thm:acc_pgm_stop:cont}
    \end{enumerate}
\end{proposition}
\begin{proof}
    \zcref[S]{thm:acc_pgm_stop:equivalence}: From~\zcref{eq:acc_pgm_opt_triv} and the fact that~$\theta^\acc_\alpha(x, y) = \varphi^\acc_\alpha(p^\acc_\alpha(x, y); x, y)$, the equivalence between~\zcref[noname]{thm:acc_pgm_stop:equivalence:sol,thm:acc_pgm_stop:equivalence:fun} is apparent.
    Let us show that~\zcref[noname]{thm:acc_pgm_stop:equivalence:Pareto,thm:acc_pgm_stop:equivalence:sol} are equivalent.
    When~$y$ is weakly Pareto optimal, we can immediately see from \zcref{thm:pgm_stop} that~$p^\acc_\alpha(x, y) = p_\alpha(y) = y$ by letting~$x = y$.
    Conversely, suppose that~$p^\acc_\alpha(x, y) = y$ for some~$x \in \setR^n$.
    Let~$z \in \setR^n$ and~$\beta \in (0, 1)$.
    The optimality of~$p^\acc_\alpha(x, y) = y$ for~\zcref{eq:acc_pgm_subprob} gives
    \begin{multline}
        \max_{i = 1, \dots, m} [ F_i(y) - F_i(x) ] \le \varphi^\acc_\alpha(y + \beta (z - y); x, y) \\
        = \max_{i = 1, \dots, m} [ \innerp{\nabla f_i(y)}{\beta (z - y)} + g_i(y + \beta (z - y)) + f_i(y) - F_i(x) ] \\
        + \frac{1}{2 \alpha} \norm*{\beta (z - y)}_2^2
        .\end{multline}
    Thus, from the convexity of~$f_i$, we get
    \begin{equation}
        \max_{i = 1, \dots, m}[ F_i(y) - F_i(x) ] \le \max_{i = 1, \dots, m}[ F_i(y + \beta (z - y)) - F_i(x) ] + \frac{1}{2 \alpha} \norm*{\beta (z - y)}_2^2
        .\end{equation}
    Moreover, the convexity of~$F_i$ yields
    \begin{multline}
        \max_{i = 1, \dots, m}[ F_i(y) - F_i(x) ] \\
        \begin{split}
            &\le \max_{i = 1, \dots, m}[ \beta F_i(z) + (1 - \beta) F_i(y) - F_i(x) ] + \frac{1}{2 \alpha} \norm*{\beta (z - y) }_2^2 \\
            &\le \beta \max_{i = 1, \dots, m}[  F_i(z) - F_i(y) ] + \max_{i = 1, \dots, m}[ F_i(y) - F_i(x) ] + \frac{1}{2 \alpha} \norm*{\beta (z - y) }_2^2
            .\end{split}
    \end{multline}
    Therefore, we get
    \begin{equation}
        \max_{i = 1, \dots, m}[  F_i(z) - F_i(y)  ] \ge - \frac{\beta}{2 \alpha} \norm*{z - y }_2^2
        .\end{equation}
    Taking~$\beta \searrow 0$, we obtain
    \begin{equation}
        \max_{i = 1, \dots, m}[  F_i(z) - F_i(y)  ] \ge 0
        ,\end{equation}
    which implies the weak Pareto optimality of~$y$.

    \zcref[S]{thm:acc_pgm_stop:cont}:
    Let~$\Omega$ be a bounded subset of~$\setR^n$ and take~$\hat{x}, \hat{y}, \check{x}, \check{y} \in \Omega$.
    Adding~\zcref{eq:acc_pgm_opt} with~$(x, y, z) \coloneqq (\hat{x}, \hat{y}, p^\acc_\alpha(\check{x}, \check{y})), (\check{x}, \check{y}, p^\acc_\alpha(\hat{x}, \hat{y}))$ gives
    \begin{align}
        \MoveEqLeft \frac{1}{\alpha} \innerp{p^\acc_\alpha(\hat{x}, \hat{y}) - p^\acc_\alpha(\check{x}, \check{y}) - (\hat{y} - \check{y})}{p^\acc_\alpha(\hat{x}, \hat{y}) - p^\acc_\alpha(\check{x}, \check{y})}                                           \\
        \le{} & \sum_{i = 1}^{m} \lambda_i(\hat{x}, \hat{y}) [\innerp{\nabla f_i(\hat{y})}{p^\acc_\alpha(\check{x}, \check{y}) - p^\acc_\alpha(\hat{x}, \hat{y})} + g_i(p^\acc_\alpha(\check{x}, \check{y})) - g_i(p^\acc_\alpha(\hat{x}, \hat{y}))]         \\
              & + \sum_{i = 1}^{m} \lambda_i(\check{x}, \check{y}) [\innerp{\nabla f_i(\check{y})}{p^\acc_\alpha(\hat{x}, \hat{y}) - p^\acc_\alpha(\check{x}, \check{y})} + g_i(p^\acc_\alpha(\hat{x}, \hat{y})) - g_i(p^\acc_\alpha(\check{x}, \check{y}))] \\
        \le{} & \sum_{i = 1}^{m} \lambda_i(\hat{x}, \hat{y}) [\innerp{\nabla f_i(\hat{y})}{\hat{y} - p^\acc_\alpha(\hat{x}, \hat{y})} - g_i(p^\acc_\alpha(\hat{x}, \hat{y})) - f_i(\hat{y}) + F_i(\hat{x})]                                                  \\
              & + \sum_{i = 1}^{m} \lambda_i(\check{x}, \check{y}) [\innerp{\nabla f_i(\check{y})}{\check{y} - p^\acc_\alpha(\check{x}, \check{y})} - g_i(p^\acc_\alpha(\check{x}, \check{y})) - f_i(\check{y}) + F_i(\check{x})]                            \\
              & + \sum_{i = 1}^{m} \lambda_i(\hat{x}, \hat{y}) [\innerp{\nabla f_i(\hat{y})}{p^\acc_\alpha(\check{x}, \check{y}) - \hat{y}} + g_i(p^\acc_\alpha(\check{x}, \check{y})) + f_i(\hat{y}) - F_i(\hat{x})]                                        \\
              & + \sum_{i = 1}^{m} \lambda_i(\check{x}, \check{y}) [\innerp{\nabla f_i(\check{y})}{p^\acc_\alpha(\hat{x}, \hat{y}) - \check{y}} + g_i(p^\acc_\alpha(\hat{x}, \hat{y})) + f_i(\check{y}) - F_i(\check{x})]
        .\end{align}
    Since~$\lambda_j(x, y)$ for~$j \in \mathcal{I}_\alpha(x)$ with~$\mathcal{I}_\alpha$ given by~\zcref{eq:acc_pgm_active}, we get
    \begin{align}
        \MoveEqLeft \frac{1}{\alpha} \innerp{p^\acc_\alpha(\hat{x}, \hat{y}) - p^\acc_\alpha(\check{x}, \check{y}) - (\hat{y} - \check{y})}{p^\acc_\alpha(\hat{x}, \hat{y}) - p^\acc_\alpha(\check{x}, \check{y})}            \\
        \le{} & \min_{i = 1, \dots, m} [\innerp{\nabla f_i(\hat{y})}{\hat{y} - p^\acc_\alpha(\hat{x}, \hat{y})} - g_i(p^\acc_\alpha(\hat{x}, \hat{y})) - f_i(\hat{y}) + F_i(\hat{x})]                                         \\
              & + \min_{i = 1, \dots, m} [\innerp{\nabla f_i(\check{y})}{\check{y} - p^\acc_\alpha(\check{x}, \check{y})} - g_i(p^\acc_\alpha(\check{x}, \check{y})) - f_i(\check{y}) + F_i(\check{x})]                       \\
              & + \sum_{i = 1}^{m} \lambda_i(\hat{x}, \hat{y}) [\innerp{\nabla f_i(\hat{y})}{p^\acc_\alpha(\check{x}, \check{y}) - \hat{y}} + g_i(p^\acc_\alpha(\check{x}, \check{y})) + f_i(\hat{y}) - F_i(\hat{x})]         \\
              & + \sum_{i = 1}^{m} \lambda_i(\check{x}, \check{y}) [\innerp{\nabla f_i(\check{y})}{p^\acc_\alpha(\hat{x}, \hat{y}) - \check{y}} + g_i(p^\acc_\alpha(\hat{x}, \hat{y})) + f_i(\check{y}) - F_i(\check{x})]     \\
        \le{} & \sum_{i = 1}^{m} \lambda_i(\check{x}, \check{y}) [\innerp{\nabla f_i(\hat{y})}{\hat{y} - p^\acc_\alpha(\hat{x}, \hat{y})} - g_i(p^\acc_\alpha(\hat{x}, \hat{y})) - f_i(\hat{y}) + F_i(\hat{x})]               \\
              & + \sum_{i = 1}^{m} \lambda_i(\hat{x}, \hat{y}) [\innerp{\nabla f_i(\check{y})}{\check{y} - p^\acc_\alpha(\check{x}, \check{y})} - g_i(p^\acc_\alpha(\check{x}, \check{y})) - f_i(\check{y}) + F_i(\check{x})] \\
              & + \sum_{i = 1}^{m} \lambda_i(\hat{x}, \hat{y}) [\innerp{\nabla f_i(\hat{y})}{p^\acc_\alpha(\check{x}, \check{y}) - \hat{y}} + g_i(p^\acc_\alpha(\check{x}, \check{y})) + f_i(\hat{y}) - F_i(\hat{x})]         \\
              & + \sum_{i = 1}^{m} \lambda_i(\check{x}, \check{y}) [\innerp{\nabla f_i(\check{y})}{p^\acc_\alpha(\hat{x}, \hat{y}) - \check{y}} + g_i(p^\acc_\alpha(\hat{x}, \hat{y})) + f_i(\check{y}) - F_i(\check{x})]
        .\end{align}
    Thus, quick calculations imply
    \begin{align}
        \MoveEqLeft \frac{1}{\alpha} \norm*{p^\acc_\alpha(\hat{x}, \hat{y}) - p^\acc_\alpha(\check{x}, \check{y})}_2^2 \le \frac{1}{\alpha} \innerp*{p^\acc_\alpha(\hat{x}, \hat{y}) - p^\acc_\alpha(\check{x}, \check{y})}{\hat{y} - \check{y}}  \\
        \le{} & \sum_{i = 1}^{m} \lambda_i(\hat{x}, \hat{y}) [\innerp{\nabla f_i(\hat{y}) - \nabla f_i(\check{y})}{p^\acc_\alpha(\check{x}, \check{y})} + f_i(\hat{y}) - f_i(\check{y}) - (F_i(\hat{x}) - F_i(\check{x}))]                        \\
              & + \sum_{i = 1}^{m} \lambda_i(\check{x}, \check{y}) [\innerp{\nabla f_i(\check{y}) - \nabla f_i(\hat{y})}{p^\acc_\alpha(\hat{x}, \hat{y})} + f_i(\check{y}) - f_i(\hat{x}) - (F_i(\check{x}) - F_i(\hat{x}))]                      \\
              & + \sum_{i = 1}^{m} \left[ \lambda_i(\check{x}, \check{y}) - \lambda_i(\hat{x}, \hat{y}) \right] \left[ \innerp{\nabla f_i(\hat{y}) - \nabla f_i(\check{y})}{\hat{y}} + \innerp{\nabla f_i(\check{y})}{\hat{y} - \check{y}}\right]
        .\end{align}
    When~$(\hat{x}, \hat{y}) \to (\check{x}, \check{y})$, the right-hand side tends to zero, so~$p^\acc_\alpha$ and~$\theta^\acc_\alpha$ are continuous.
    Moreover, assume that~$\nabla f_i$ is locally Lipschitz continuous for~$i = 1, \dots, m$.
    Since~$f_i, g_i, F_i$ are also locally Lipschitz, the above inequality shows that~$p^\acc_\alpha$ is locally H\"older continuous with exponent~$1 / 2$.

    On the other hand, the definition~\zcref{eq:acc_pgm_optimal_function} of~$p^\acc_\alpha$ and~$\theta^\acc_\alpha$ gives
    \begin{align}
        \MoveEqLeft \theta^\acc_\alpha(\hat{x}, \hat{y}) - \theta^\acc_\alpha(\check{x}, \check{y}) \le \varphi^\acc_\alpha(p^\acc_\alpha(\check{x}, \check{y}); \hat{x}, \hat{y}) - \varphi^\acc_\alpha(p^\acc_\alpha(\check{x}, \check{y}); \check{x}, \check{y}) \\
        ={}   & \max_{i = 1, \dots, m} [\innerp{\nabla f_i(\hat{y})}{p^\acc_\alpha(\check{x}, \check{y}) - \hat{y}} + g_i(p^\acc_\alpha(\check{x}, \check{y})) + f_i(\hat{y}) - F_i(\hat{x})]                                                                       \\
              & - \max_{i = 1, \dots, m} [\innerp{\nabla f_i(\check{y})}{p^\acc_\alpha(\check{x}, \check{y}) - \check{y}} + g_i(p^\acc_\alpha(\check{x}, \check{y})) + f_i(\check{y}) - F_i(\check{x})]                                                             \\
              & + \frac{1}{2 \alpha} \Big[ \norm{p^\acc_\alpha(\check{x}, \check{y}) - \hat{y}}_2^2 - \norm{p^\acc_\alpha(\check{x}, \check{y}) - \check{y}}_2^2 \Big]                                                                                              \\
        \le{} & \max_{i = 1, \dots, m}
        \begin{multlined}[t]
            [ \innerp{\nabla f_i(\check{y})}{\check{y} - \hat{y}} + \innerp{\nabla f_i(\hat{y}) - \nabla f_i(\check{y})}{p^\acc_\alpha(\check{x}, \check{y}) - \hat{y}} \\
                + f_i(\hat{y}) - f_i(\check{y}) - F_i(\hat{x}) + F_i(\check{x}) ]
        \end{multlined}                                                                                                                                \\
              & + \frac{1}{2 \alpha} \innerp{2 p^\acc_\alpha(\check{x}, \check{y}) - \hat{y} - \check{y}}{\check{y} - \hat{y}}                                                                                                                                      \\
        \le{} & \max_{i = 1, \dots, m} \norm{\nabla f_i(\check{y})}_2 \norm{\hat{y} - \check{y}}_2 + \norm{\hat{y} - p^\acc_\alpha(\check{x}, \check{y})}_2 \max_{i = 1, \dots, m} \norm{\nabla f_i(\hat{y}) - \nabla f_i(\check{y})}_2                             \\
              & + \max_{i = 1, \dots, m} \abs{f_i(\hat{y}) - f_i(\check{y})} + \max_{i = 1, \dots, m} \abs{F_i(\hat{x}) - F_i(\check{x})}                                                                                                                           \\
              & + \frac{1}{2 \alpha} \norm{2 p^\acc_\alpha(\check{x}, \check{y}) - \hat{y} - \check{y}}_2 \norm{\hat{y} - \check{y}}_2
        ,\end{align}
    where the second inequality follows from~\zcref{eq:min_diff_ineq}, and the third inequality comes from the Cauchy-Schwarz inequalities.
    Since the above inequality holds even if we interchange~$(\hat{x}, \hat{y})$ and~$(\check{x}, \check{y})$, we can show the Lipschitz continuity of~$\theta^\acc_\alpha$ on~$\Omega$ in the same way as in the previous paragraph.
\end{proof}
\zcref[S]{thm:acc_pgm_stop} suggests that we can use~$\norm*{p^\acc_\alpha(x, y) - y}_\infty < \varepsilon$ for some~$\varepsilon > 0$ as a stopping criterion.
Now, we state below the proposed algorithm.

\begin{algorithm}[hbtp]
    \caption{Accelerated proximal gradient method with general stepsizes for~\zcref{eq:MOO}}
    \zlabel{alg:acc_pgm_MO}
    \begin{algorithmic}[1]
        \Require $x^0 = y^1 \in \dom(F), a \in [0, 1), b \in [a^2 / 4, 1 / 4], \varepsilon > 0$.
        \Ensure $x^\ast$: A weakly Pareto optimal point
        \State $k \gets 1$
        \State $t_1 \gets 1$ \zlabel{alg:acc_pgm_MO:mom_fac_ini}
        \Loop
        \State $x^k \gets p^\acc_\alpha(x^{k - 1}, y^k)$ for some stepsize~$\alpha > 0$
        \If{$\norm{x^k - y^k}_\infty < \varepsilon$}
        \State \Return $x^k$
        \EndIf
        \State $t_{k + 1} \gets \sqrt{t_k^2 - a t_k + b} + 1/2$ \zlabel{alg:acc_pgm_MO:mom_fac_rr}
        \State $\gamma_k \gets (t_k - 1) / t_{k + 1}$ \zlabel{alg:acc_pgm_MO:mom_fac2}
        \State $y^{k + 1} \gets x^k + \gamma_k (x^k - x^{k - 1})$ \zlabel{alg:acc_pgm_MO:seq2}
        \State $k \gets k + 1$
        \EndLoop
    \end{algorithmic}
\end{algorithm}

The sequence~$\set{t_k}$ defined in~\zcref{alg:acc_pgm_MO:mom_fac_ini,alg:acc_pgm_MO:mom_fac_rr} of \zcref{alg:acc_pgm_MO} generalizes the well-known momentum factors in single-objective accelerated methods.
For example, when~$a = 0$ and~$b = 1 / 4$, they coincide with the one in \zcref{alg:acc_pgm_MO} and the original FISTA~\cite{Nesterov1983,Beck2009} ($t_1 = 1$ and~$t_{k + 1} = (1 + \sqrt{1 + 4 t_k^2}) / 2$).
Moreover, if~$b = a^2 / 4$, then~$\set{t_k}$ has the general term~$t_k = (1 - a) k / 2 + (1 + a) / 2$, which corresponds to the one used in~\cite{Chambolle2015,Su2016,Attouch2016,Attouch2018}.
This means that our generalization allows finer algorithm tuning by varying~$a$ and~$b$.
We show below some properties of~$\set*{t_k}$ and~$\set*{\gamma_k}$.
\begin{lemma} \zlabel{thm:mom_fac}
    Let~$\set{t_k}$ and~$\set{\gamma_k}$ be defined by \zcref{alg:acc_pgm_MO:mom_fac_ini,alg:acc_pgm_MO:mom_fac_rr,alg:acc_pgm_MO:mom_fac2} in \zcref{alg:acc_pgm_MO} for arbitrary~$a \in [0, 1)$ and~$b \in [a^2 / 4, 1 / 4]$.
    Then, the following inequalities hold for all~$k \ge 1$.
    \begin{enumerate}
        \item $t_{k + 1} \ge t_k + \dfrac{1 - a}{2}$ and~$t_k \ge \dfrac{1 - a}{2} k + \dfrac{1 + a}{2}$; \zlabel{thm:mom_fac:ge}
        \item $t_{k + 1} \le t_k + \dfrac{1 - a + \sqrt{4b - a^2}}{2}$ and~$t_k \le \dfrac{1 - a + \sqrt{4b - a^2}}{2} (k - 1) + 1 \le k$; \zlabel{thm:mom_fac:le}
        \item $t_k^2 - t_{k + 1}^2 + t_{k + 1} = a t_k - b + \dfrac{1}{4} \ge a t_k$; \zlabel{thm:mom_fac:gratio}
        \item $0 \le \gamma_k \le \dfrac{k - 1}{k + 1 / 2}$; \zlabel{thm:mom_fac:fac2}
        \item $1 - \gamma_k^2 \ge \dfrac{1}{t_k}$. \zlabel{thm:mom_fac:fac2_le1}
    \end{enumerate}
\end{lemma}
\begin{proof}
    \zcref[S]{thm:mom_fac:ge}:
    From the definition of~$\set{t_k}$, we have
    \begin{equation} \label{eq:t cs}
        t_{k + 1} = \sqrt{t_k^2 - a t_k + b} + \frac{1}{2}
        = \sqrt{\left( t_k - \frac{a}{2} \right)^2 + \left( b - \frac{a^2}{4} \right)} + \frac{1}{2}
        .\end{equation}
    Since~$b \ge a^2 / 4$, we get
    \begin{equation}
        t_{k + 1} \ge \abs*{t_k - \frac{a}{2}} + \frac{1}{2}
        .\end{equation}
    Since~$t_1 = 1 \ge a / 2$, we can quickly see that~$t_k \ge a / 2$ for any~$k$ by induction.
    Thus, we have
    \begin{equation}
        t_{k + 1} \ge t_k + \frac{1 - a}{2}
        .\end{equation}
    Applying the above inequality recursively, we obtain
    \begin{equation}
        t_k \ge \frac{1 - a}{2} (k - 1) + t_1 = \frac{1 - a}{2} k + \frac{1 + a}{2}
        .\end{equation}

    \zcref[S]{thm:mom_fac:le}:
    From~\zcref{eq:t cs} and the relation~$\sqrt{\beta_1 + \beta_2} \le \sqrt{\beta_1} + \sqrt{\beta_2}$ with~$\beta_1, \beta_2 \ge 0$, we get the first inequality.
    Using it recursively, it follows that
    \begin{equation}
        t_k \le \frac{1 - a + \sqrt{4 b - a^2}}{2} (k - 1) + t_1 = \frac{1 - a + \sqrt{4 b - a^2}}{2} (k - 1) + 1
        .\end{equation}
    Since~$a \in [0, 1), b \in [a^2 / 4, 1 / 4]$, we observe that
    \begin{equation}
        \frac{1 - a + \sqrt{4 b - a^2}}{2} \le \frac{1 - a + \sqrt{1 - a^2}}{2} \le 1
        .\end{equation}
    Hence, the above two inequalities lead to the desired result.

    \zcref[S]{thm:mom_fac:gratio}:
    An easy computation shows that
    \begin{equation}
        \begin{split}
            t_k^2 - t_{k + 1}^2 + t_{k + 1} &= t_k^2 - \left[ \sqrt{t_k^2 - a t_k + b} + \frac{1}{2} \right]^2 + \sqrt{t_k^2 - a t_k + b} + \frac{1}{2} \\
            &= a t_k - b + \frac{1}{4} \ge a t_k
            ,\end{split}
    \end{equation}
    where the inequality holds since~$b \le 1 / 4$.

    \zcref[S]{thm:mom_fac:fac2}:
    The first inequality is clear from the definition of~$\gamma_k$ since \zcref{thm:mom_fac:ge} yields~$t_k \ge 1$.
    Again, the definition of~$\gamma_k$ and \zcref{thm:mom_fac:ge} give
    \begin{equation}
        \gamma_k = \frac{t_k - 1}{t_{k + 1}} \le \frac{t_k - 1}{t_k + (1 - a) / 2} = 1 - \frac{3 - a}{2 t_k + 1 - a}
        .\end{equation}
    Combining with the \zcref{thm:mom_fac:le}, we get
    \begin{equation} \label{eq:gamma}
        \begin{split}
            \gamma_k &\le 1 - \frac{3 - a}{\left(1 - a + \sqrt{4 b - a^2} \right)(k - 1) + 3 - a} \\
            &= \frac{\left( 1 - a + \sqrt{4 b - a^2} \right) (k - 1)}{\left(1 - a + \sqrt{4 b - a^2} \right)(k - 1) + 3 - a} \\
            &= \frac{k - 1}{k - 1 + (3 - a) / \left( 1 - a + \sqrt{4 b - a^2} \right)}
            .\end{split}
    \end{equation}
    On the other hand, it follows that
    \begin{equation} \label{eq:min a b}
        \min_{a \in [0, 1), b \in [a^2 / 4, 1 / 4]} \frac{3 - a}{1 - a + \sqrt{4 b - a^2}} = \min_{a \in [0, 1)} \frac{3 - a}{1 - a + \sqrt{1 - a^2}} = \frac{3}{2}
        ,\end{equation}
    where the second equality follows from the monotonic non-decreasing property implied by
    \begin{equation}
        \odv*{\left(\frac{3 - a}{1 - a + \sqrt{1 - a^2}}\right)}{a} = \frac{2 \sqrt{1 - a^2} + 3a - 1}{\left( \sqrt{1 - a^2} - a + 1 \right)^2 \sqrt{1 - a^2}} > 0 \forallcondition{a \in [0, 1)}
        .\end{equation}
    Combining~\zcref{eq:gamma,eq:min a b}, we obtain~$\gamma_k \le (k - 1) / (k + 1 / 2)$.

    \zcref[S]{thm:mom_fac:fac2_le1}:
    \zcref{thm:mom_fac:ge} implies that~$t_{k + 1} > t_k \ge 1$.
    Thus, the definition of~$\gamma_k$ implies that
    \begin{equation}
        1 - \gamma_k^2 = 1 - \left( \frac{t_k - 1}{t_{k + 1}} \right)^2 \ge 1 - \left( \frac{t_k - 1}{t_k} \right)^2
        = \frac{2 t_k - 1}{t_k^2} \ge \frac{2 t_k - t_k}{t_k^2} = \frac{1}{t_k}
        .\end{equation}
\end{proof}

We end this section by noting some remarks about the proposed algorithm.
\begin{remark} \zlabel{rem: acc-pgm}
    \begin{enumerate}
        \item Since~$x \in \dom(F)$ implies~$p^\acc_\alpha(x, y) \in \dom(F)$, every~$x^k$ computed by the above algorithm is in~$\dom(F)$.
              However,~$y^k$ is not necessarily in~$\dom(F)$.
        \item Since $y^1 = x^0$, it follows from~\zcref{eq:acc_pgm_opt_triv} that
              \begin{equation} \label{eq:subprob ineq}
                  \theta^\acc_\alpha(x^0, y^1) \le 0
                  ,\end{equation}
              but the inequality~$\theta^\acc_\alpha(x^{k - 1}, y^k) \le 0$ does not necessarily hold for~$k \ge 2$.
        \item When~$m = 1$, we can remove the term~$f_i(y) - F_i(x)$ from the subproblem~\zcref{eq:acc_pgm_subprob}, so \zcref{alg:acc_pgm_MO} corresponds to the Fast Iterative Shrinkage-Thresholding Algorithm (FISTA)~\cite{Beck2009} for single-objective optimization.
        \item \zcref{alg:acc_pgm_MO} induces the accelerated versions of first-order algorithms such as the steepest descent~\cite{Fliege2000}, proximal point~\cite{Bonnel2005}, and projected gradient methods~\cite{Grana-Drummond2004}.
    \end{enumerate}
\end{remark}

\end{document}
