\documentclass[../main]{subfiles}

\begin{document}
\section{Introduction}
This chapter develops the accelerated proximal gradient method for the unconstrained comvex composite multi-objective optimization, i.e.,~\cref{eq:MOO} with~\cref{eq:composite_MO},~$f_i$ is convex, and~$C = \setR^n$.

There are many studies related to the acceleration of single-objective first-order methods.
After being established by Nesterov~\cite{Nesterov1983}, researchers developed various accelerated schemes.
In particular, the Fast Iterative Shrinkage-Thresholding Algorithm (FISTA)~\cite{Beck2009}, an accelerated version of the proximal gradient method, has contributed to a wide range of research fields, including image and signal processing.
However, in the multi-objective case, the studies associated with accelerated algorithms are still insufficient.
In 2020, El Moudden and El Mouatasim~\cite{ElMoudden2020} proposed an accelerated diagonal steepest descent method for multi-objective optimization, a natural extension of Nesterov's accelerated method for single-objective problems.
They proved the global convergence rate of the algorithm ($O(1 / k^2)$) under the assumption that the sequence of the Lagrange multipliers of the subproblems is eventually fixed.
Nevertheless, this assumption is restrictive because it indicates that the approach is essentially the same as the (single-objective) Nesterov's method, only applied to the minimization of a weighted sum of the objective functions.

Here, we propose a genuine accelerated proximal gradient method for multi-objective optimization.
As it is usual, in each iteration, we solve a convex (scalar-valued) subproblem.
While the accelerated and non-accelerated algorithms solve the same subproblem in the single-objective case, the subproblem of our accelerated method has terms that are not included in the non-accelerated version.
However, we can ignore these terms in the single-objective case, and thus we can regard our proposed method as a generalization of FISTA.
Moreover, under more natural assumptions, we prove the proposed method's global convergence rate ($O(1/k^2)$) by using a merit function~\cref{eq:u_0} to measure the complexity.

Furthermore, having the practical computational efficiency in mind, we derive a dual of the subproblem, which is convex and differentiable.
Such a dual problem turns out to be easier to solve than the original one, especially when the number of objective functions is smaller than the dimension of the decision variables.
We can also reconstruct the original subproblem's solution directly from the dual optimum.
In addition, we implement the whole algorithm using this dual problem and confirm its effectiveness with numerical experiments.

The outline of this paper is as follows.
We present the accelerated proximal gradient method for multi-objective optimization in \cref{sec:acc_pgm:algorithm} and analyze its~$O(1 / k^2)$ convergence rate in \cref{sec:acc_pgm:rate}.
Moreover, \cref{sec:acc_pgm:convergence} demonstrates the convergence of the iterates.
Finally, we report some numerical results for test problems in \cref{sec:acc_pgm:experiments}, demonstrating that the proposed method is faster than the one without acceleration.

\end{document}
