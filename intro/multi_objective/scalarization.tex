\documentclass[../../main]{subfiles}

\begin{document}
\subsection{Scalarization approach}
The \emph{scalarization approach}~\cite{Gass1955,Geoffrion1968,Zadeh1963} is one of the most popular strategies for multi-objective problems.
It converts the original multi-objective problem into a parameterized scalar-valued problem.

Let us now introduce the \emph{weighted sum method}~\cite{Zadeh1963}, one of the most well-known scalarization techniques.
It scalarizes~\zcref{eq:MOO} with the weight vector~$w \coloneqq (w_1, \dots, w_m)^\T \in \setR^m$ as follows:
\begin{equation} \label{eq:weighted}
    \min_{x \in \setR^n} \quad \innerp{w}{F(x)}
    ,\end{equation}
where
\begin{equation}
    w \ge 0 \eqand \sum_{i = 1}^{m} w_i = 1
    .\end{equation}
When~$F$ is convex, for every Pareto optimal solution~$x^\ast$ of~\zcref{eq:MOO}, there exists~$w$ such that~$x^\ast$ is the solution of~\zcref{eq:weighted}~\cite{Miettinen1998}.
However, it may be challenging to choose a \emph{good} weight in advance.
Moreover, if~$F$ is non-convex, there may be Pareto optimal solutions that are not the solutions of~\zcref{eq:weighted} for any~$w$, and some~$w$ may make~\zcref{eq:weighted} unbounded.

\end{document}
