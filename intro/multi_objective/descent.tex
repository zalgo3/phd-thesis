\documentclass[../../main]{subfiles}

\begin{document}
\subsection{Descent methods} \label{sec:intro:multi_objective:descent}
\emph{Descent methods}~\cite{Fukuda2014} are iterative algorithms that decrease the objective function values at each iteration.
They do not require \emph{a priori} parameters selection like scalarization, and unlike heuristics, we can analyze their global convergence property under reasonable assumptions.
All algorithms proposed in this thesis are part of the descent methods.
Below we provide typical descent methods for~\cref{eq:MOO}.

\begin{example} \label{ex:descent}
    \begin{description}
        \item[The steepest descent method~\cite{Fliege2000}] Consider a smooth unconstrained multi-objective optimization, i.e.,~$C = \setR^n$ and each~$F_i$ is differentiable in~\cref{eq:MOO}.
            Then, the steepest descent method updates~$\set{x^k}$ by the following operations:
            \begin{align} \label{eq:steepest_descent}
                d^k &\coloneqq \argmin_{d \in \setR^n} \left[ \max_{i = 1, \dots, m} \innerp*{\nabla F_i(x^k)}{d} + \frac{1}{2 \lambda_k} \norm{d}_2^2 \right], \\
                x^{k + 1} &\coloneqq x^k + s_k d^k
            \end{align}
            with~$\lambda_k > 0$ and~$s_k > 0$.
            When~$m = 1$, we have~$d^k = - \lambda_k \nabla F_1(x)$, which is the steepest descent direction for the scalar optimization~\cite{Cauchy1847}.
        \item[The projected gradient method~\cite{Grana-Drummond2004,Fukuda2013}] For a convex-constrained smooth multi-objective optimization, i.e.,~$C \subseteq \setR^n$ is non-empty, closed, and convexand every~$F_i$ is differentiable in~\cref{eq:MOO}, we can use the projected gradient method described by
            \[ \label{eq:proj_grad}
                \begin{aligned} 
                    z^k &\coloneqq \argmin_{z \in C} \left[ \max_{i = 1, \dots, m} \innerp*{\nabla F_i(x^k)}{z - x^k} + \frac{1}{2 \lambda_k} \norm*{z - x^k}_2^2 \right], \\
                    x^{k + 1} &\coloneqq x^k + s_k \left(z^k - x^k \right)
                \end{aligned}
            \] 
            with~$\lambda_k > 0$ and~$s_k > 0$.
            When~$m = 1$,~\cref{eq:proj_grad} reduces to the projected gradient method for scalar optimization~\cite{Polyak1963,Goldstein1964,Goldstein1967,McCormick1969}, i.e.,
            \[
                \begin{aligned}
                    z^k &\coloneqq \proj_C (x^k - \lambda_k \nabla F_1(x^k)), \\
                    x^{k + 1} &\coloneqq x^k + s_k \left( z^k - x^k \right)
                ,\end{aligned}
            \] 
            where~$\proj_C$ denotes the projection onto~$C$ given by
            \[ \label{eq:proj}
                \proj_C(x) \coloneqq \argmin_{z \in C} \norm{z - x}_2
            .\] 
            Moreover, when~$C = \setR^n$,~\cref{eq:proj_grad} amounts to the steepest descent method~\cref{eq:steepest_descent}.
        \item[The projected subgradient method~\cite{Bello-Cruz2013}]
            Focus on a convex-constrained, non-smooth, and convex multi-objective optimization, i.e.,~$C$ is a non-empty, closed, and convex subset of~$\setR^n$, and each~$F_i$ is convex and non-differentiable in~\cref{eq:MOO}.
            The subgradient method requires an exogenous sequence~$\set{\beta_k}$ satisfying
            \[
                \beta_k > 0, \quad \sum_{k = 0}^{\infty} \beta_k = \infty, \eqand \sum_{k = 0}^{\infty} \beta_k^2 < \infty 
            \] 
            and generates~$\set{x^k}$ by
            \[
                    x^{k + 1} \coloneqq \argmin_{z \in C} \left[ \frac{1}{2} \norm{z - x^k}_2^2 + \frac{\beta_k}{\eta_k} \max_{i = 1, \dots, m} \innerp{\xi^k_i}{z - x^k} \right] 
            ,\] 
            where~$\xi^k_i \in \partial F_i(x^k)$ and
            \[
                \eta_k \coloneqq \max_{i = 1, \dots, m} \norm{\xi_i}
            .\] 
            When~$m = 1$, this step represents the projected subgradient method~\cite{Polyak1967,Polyak1969,Shor1985,Alber1998,Alber2001} for scalar optimization:
            \[
                x^{k + 1} \coloneqq \proj_C\left( x^k - \frac{\beta_k}{\eta_k} \xi^k_1 \right) 
            .\] 
    \end{description}
\end{example}

\end{document}
