\documentclass[../../main]{subfiles}

\begin{document}
\section{Composite optimization} \zlabel{sec:intro:composite}
\emph{Composite optimization} has the following structure:
\begin{equation} \label{eq:composite}
    \min_{x \in \setR^n} \quad F(x) \coloneqq f(x) + g(x)
,\end{equation} 
where~$f \colon \setR^n \to \setR$ is $L_f$-smooth with some~$L_f > 0$, and~$g \colon \setR^n \to \setR \cup \set{\infty}$ is closed, proper, and convex.
When~$f$ is convex, we call~\zcref{eq:composite} \emph{convex composite}.
This model has many applications, particularly in machine learning.
In detail,~$f$ and~$g$ often represent the loss function and the regularization term, respectively.
We list below some typical examples with the structure~\zcref{eq:composite}.

\begin{example} \zlabel{ex:composite}
    \begin{description}
        \item[Smooth unconstrained minimization] If~$g = 0$,~\zcref{eq:composite} reduces to the unconstrained smooth minimization
            \begin{equation}
                \min_{x \in \setR^n} \quad f(x)
            ,\end{equation} 
            where~$f \colon \setR^n \to \setR$ is~$L_f$-smooth.
        \item[Convex-constrained smooth minimization] If~$g$ is an indicator function of a non-empty, closed, and convex set~$C$, i.e.,
            \begin{equation} \label{eq:indicator}
                g(x) = \indicator_C(x) \coloneqq
                \begin{cases}
                    0 & x \in C, \\
                    \infty & \otherwise,
                \end{cases}
            \end{equation} 
            then~\zcref{eq:composite} amounts to the convex-constrained smooth minimization
            \begin{equation}
                \min_{x \in C} \quad f(x)
            \end{equation} 
            with an~$L$-smooth function~$f$.
        \item[$\ell_1$-regularization] If~$g(x) \coloneqq \tau \norm{x}_1$ for some~$\tau > 0$,~\zcref{eq:composite} reduces to the~$\ell_1$-regularizaiton
            \begin{equation}
                \min_{x \in C} \quad f(x) + \tau \norm{x}_1
            \end{equation} 
            with~$f$ being~$L_f$-smooth.
    \end{description}
\end{example}

\subfile{prox_grad}

\subfile{acc_prox_grad}

\end{document}
