\documentclass[../../main]{subfiles}

\begin{document}
\subsection{The proximal gradient method} \zlabel{sec:intro:composite:pgm}
The \emph{proximal gradient method}~\cite{Fukushima1981} is one of the most common algorithms for solving~\zcref{eq:composite}.
For a given~$x^0 \in \interior (\dom (F))$, it recursively update~$\set*{x^k} $ by
\begin{equation}
    x^{k + 1} = \prox_{\alpha_k g} \left( x^k - \alpha_k \nabla f\left(x^k\right) \right) 
,\end{equation} 
where~$\prox$ is the \emph{proximal operator}, which we will define in~\zcref{eq:prox}.
If we can estimate the Lipschitz constant~$L_f$, we can use a constant stepsize~$\alpha_k \in (0, 1 / L_f]$.
Otherwise, we can determine~$\alpha_k$ in each iteration by backtracking.

The description of the algorithm now follows.
\begin{algorithm}[hbtp]
    \caption{The proximal gradient method}
    \zlabel{alg:pgm}
    \begin{algorithmic}[1]
        \Require $x^0 \in \interior (\dom (F)), \varepsilon > 0$
        \State $k \gets 0$
        \Repeat
        \State pick~$\alpha_k > 0$
        \State $x^{k + 1} \gets \prox_{\alpha_k g}(x^k - \alpha_k \nabla f\left(x^k\right))$
        \State $k \gets k + 1$
        \Until{$\norm{x^k - x^{k - 1}}_\infty < \varepsilon$}
        \State \Return $x^k$
    \end{algorithmic}
\end{algorithm}

With this algorithm,~$\set*{\norm{x^{k + 1} - x^k}_2}$ converges to zero with a rate of~$O\left(\sqrt{1 / k}\right)$ and every accumulation point of~$\set*{x^k}$, if it exists, is a stationary point~\cite{Beck2017}.
When~$f$ is convex,~$\set*{x^k}$ converges to the global minima~$x^\ast$, and~$\set*{F\left(x^k\right) - F(x^\ast)}$ converges to zero with a rate of~$O(1 / k)$~\cite{Beck2017}.
Moreover, when~$f$ is strongly convex,~$\set*{x^k}$ converges linearly to~$x^\ast$~\cite{Beck2017}.
Furthermore, if we assume the so-called proximal-PL condition, which we will define by~\zcref{eq:proximal_PL},~$\set*{F\left(x^k\right)}$ converges linearly to~$F(x^\ast)$~\cite{Karimi2016}.

\end{document}
