\documentclass[../main]{subfiles}

\begin{document}
\section{Motivations and contributions}
As discussed in \cref{sec:intro:multi_objective}, multi-objective optimization~\cref{eq:MOO} is an indispensable model in dealing with real-world problems, and the studies on its theories and algorithms have great significance.
On the other hand, many previous studies on multi-objective optimization, particularly on the descent methods described in \cref{sec:intro:multi_objective:descent} and the merit functions described in \cref{sec:intro:merit:MO}, have dealt with smooth problems, and there is still room for exploration for non-smooth problems.
The projected subgradient method inroduced in \cref{ex:descent} can handle non-smooth multi-objective optimization, but it may not work well for large-scale problems due to the stepsize decay.

This thesis focuses on non-smooth multi-objective optimization problems with specific structures, mainly the generalization of the composite model introduced in \cref{sec:intro:composite}, i.e.,~\cref{eq:MOO} with
\[ \label{eq:composite_MO}
    F_i(x) = f_i(x) + g_i(x) \forallcondition{i = 1, \dots, m}
,\] 
where~$f_i$ is continuously differentiable and~$g_i$ is closed, proper, and convex.
Then, we presents their theory and algorithms.
\end{document}
