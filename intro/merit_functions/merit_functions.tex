\documentclass[../../main]{subfiles}

\begin{document}
\section{Merit functions} \zlabel{sec:intro:merit}
\emph{Merit functions}~\cite{Fukushima1996} are maps that return zeros at the problems' solutions and strictly positive values otherwise.
In other words, they are the objective functions of optimization problems with the same solutions as the original problems.
Therefore, the merit functions should have the following properties:
\begin{itemize}
    \item Quick computability;
    \item Continuity;
    \item Differentiability;
    \item Optimality of the stationary points;
    \item Level-boundedness;
    \item Error-boundedness.
\end{itemize}
Moreover, as we can consider the merit functions to represent how far feasible points are from the optimal solutions, they help analyze convergence rates of optimization algorithms.

\subfile{VI}

\subfile{MO}

\end{document}
